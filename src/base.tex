\documentclass[14pt]{extarticle} %Класс позволяет использовать базовые шрифты бОльших размеров
\usepackage[utf8x]{inputenc} %кодировка файла макета utf8
\usepackage[russian]{babel} 
\usepackage[left=25mm,right=15mm,top=20mm,bottom=20mm]{geometry} %Попытка разобраться с полями страниц
\usepackage{ntheorem} %окружение для настройки теорем 
\usepackage{graphicx} %работа с рисунками
\usepackage[labelsep=period,figurewithin=none,tablewithin=none]{caption} %подписи к объектам (рисунки, таблицы)
\usepackage{listings} %работа с листингами
\usepackage{indentfirst} %отступ первого абзаца в разделе
\usepackage{enumitem} %настройка маркированных и нумерованных списков (см. примеры настройки в тексте)
\usepackage{url} %формирование ссылок на электронные источники
\usepackage{fancyhdr} %Настройка нумерации страниц
\usepackage{tocloft} %Настройка заголовка для содержания

%====================================================================
% Мои настройки
\usepackage{amsmath}
\usepackage{amssymb}
\usepackage{booktabs}
\usepackage{caption}
\usepackage{subfig}
\usepackage{algorithm}
\usepackage{algorithmic}
\usepackage{bm}
\usepackage[para,online,flushleft]{threeparttable}
%\graphicspath{ {imgs/} }
\usepackage{xifthen}

%====================================================================

%====================================================================
%Настройки макета
%----------------
%Содержимое этого блока не должно подвергаться изменению
%====================================================================
\selectlanguage{russian}
\setlength{\parindent}{1.27cm}

%---------Размеры страницы
%\textwidth=11.3cm \textheight=17cm 
%\hoffset=-30mm \voffset=-15mm

%---------Настройка подписей к таблицам
\DeclareCaptionFormat{mplain}{#1#2\par \centering #3\par}
\captionsetup[table]{format=mplain,
justification=raggedleft,%
labelsep=none,%
singlelinecheck=false,%
skip=3pt}

%---------Настройка подписей к таблицам
%\setbeamertemplate{caption}[numbered]
%\setbeamertemplate{footline}[frame number]
\newcommand{\rulesep}{\unskip\ \vrule\ }
\newcommand{\source}[1]{\caption*{\textbf{Источник}: {#1}} }

%Настройка нумерации страниц
\fancyhf{} % clear all header and footers
\renewcommand{\headrulewidth}{0pt} % remove the header rule
\rfoot{\small \thepage}
\pagestyle{fancy}

%Настройка заголовка для содержания
\renewcommand{\cfttoctitlefont}{\hfill\normalfont\large\bfseries}
\renewcommand{\cftaftertoctitle}{\hfill\thispagestyle{empty}} 

%Настройка теорем
\theoremseparator{.}

%---------Команды рубрикации--------------

%Заголовки
\makeatletter
\renewcommand{\section}{\@startsection{section}{1}%
{\parindent}{-3.5ex plus -1ex minus -.2ex}%
{2.3ex plus.2ex}{\normalfont\large\bfseries}}

\renewcommand{\subsection}{\@startsection{subsection}{2}%
{\parindent}{-3.5ex plus -1ex minus -.2ex}%
{1.5ex plus.2ex}{\normalfont\large\bfseries}}

\renewcommand{\subsubsection}{\@startsection{subsubsection}{3}%
{\parindent}{-1.5ex plus -1ex minus -.2ex}%
{0.5ex plus.2ex}{\normalfont\bfseries}}
\makeatother

%Команда уровня главы
\newcommand{\mysection}[1]{
 \newpage
 \refstepcounter{section}
 {
  \section*{Глава \thesection. #1 \raggedright }
 }
 \addcontentsline{toc}{section}{Глава \thesection. #1} 
}

%Команда уровня параграфа
\newcommand{\mysubsection}[1]{
 \refstepcounter{subsection}
 \subsection*{\thesubsection. #1}
 \addcontentsline{toc}{subsection}{\thesubsection. #1}
}

%Команда третьего уровня
\newcommand{\mysubsubsection}[1]{
\refstepcounter{subsubsection}
% \addcontentsline{toc}{subsubsection}{\thesubsubsection. #1}
\subsubsection*{#1}
}

%Оформление Приложений
\newcounter{appendix}
\newcommand{\addappendix}[1]{
 \newpage
 \refstepcounter{appendix} 
 \section*{ПРИЛОЖЕНИЕ \theappendix. \\#1}
 \addcontentsline{toc}{section}{ПРИЛОЖЕНИЕ \theappendix. #1}
}

%Команда ненумерованной главы
\newcommand{\mynonumbersection}[1]{
\newpage
{
%	\begin{center}\section*{#1}\end{center}
	\centering\section*{#1}
}
\addcontentsline{toc}{section}{#1} 
}

%--------Настройка маркированных и нумерованных списков
\setlist{itemsep=0pt,topsep=0pt}

%--------Настройка листингов программного кода
\lstloadlanguages{C,[ANSI]C++}%!настройка листинга
%Можно подключить другие языки (см документацию к пакету)

%--------Тонкая настройка листингов
\lstset{
inputencoding=utf8x,
extendedchars=false,
showstringspaces=false,
showspaces=false,
keepspaces = true,
basicstyle=\small\ttfamily,
keywordstyle=\bfseries,
tabsize=2,                      % sets default tabsize to 2 spaces
captionpos=t,                   % sets the caption-position to bottom
breaklines=true,                % sets automatic line breaking
breakatwhitespace=true,        % sets if automatic breaks should only happen at whitespace
title=\lstname,                 % show the filename of files included with \lstinputlisting;
basewidth={0.5em,0.45em},
}

%----------Настройка подписей к листингам
\renewcommand{\lstlistingname}{Листинг}

%------------Подключение стиля для оформления списка литературы
\makeatletter
\renewcommand{\@biblabel}[1]{#1.\hfill}
\makeatother
\bibliographystyle{ugost2003s}
\PrerenderUnicode{ЙЦУКЕНГШЩЗХЪЭЖДЛОРПАВЫФЯЧСМИТЬБЮйцукенгшщзхъэждлорпавыфячсмитьбю}


%----------Use frames for text
\usepackage{mdframed}
\newcommand{\myframe}[1]{
\begin{mdframed}
\begin{center}
#1
\end{center}
\end{mdframed}
}

%----------Setup images
\captionsetup{figurename=Рисунок}

\newcommand{\myimage}[4]{ % source?; width; caption; img
\begin{figure}[h]
\centering
\includegraphics[width=#2\textwidth]{#4}
\caption{#3}
\ifthenelse{\equal{#1}{}}{
}{
\source{#1}
}
\end{figure}
}

%----------Setup Math stuff
\newtheorem{definition}{Определение}[section]
\newtheorem{proof}{Доказательство}
\newtheorem{theorem}{Теорема}

\makeatletter
\renewcommand*{\ALG@name}{Алгоритм}
\makeatother

\newcommand{\mydefinition}[1]{
\myframe{
\theoremstyle{definition}
\begin{definition}
#1
\end{definition}
}}

\newcommand{\myproof}[1]{
\theoremstyle{proof}
\begin{proof}
#1
\end{proof}
}

\newcommand{\mytheorem}[1]{
\myframe{
\theoremstyle{theorem}
\begin{theorem}
#1
\end{theorem}
}}

\newcommand{\myequation}[2][]{
\begin{equation}
\ifthenelse{\equal{#1}{}}{
\begin{aligned}#2\end{aligned}
}{
\left.\begin{aligned}#2\end{aligned}\right\} #1
}
\end{equation}
}
