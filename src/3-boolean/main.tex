\input{../base.tex}

\begin{document}

%==================================================
\mynonumbersection{БУЛЕВА АЛГЕБРА}

%==================================================
\mysection{Введение}
Современные компьютеры спроектированы с использованием подходов и обозначений из области математики, называемой \textbf{алгеброй}. Алгебраисты уже более сотни лет изучают математические системы именуемые \textbf{булевыми алгебрами}. Нет ничего более простого и естественного для человеческого понимания, чем законы булевой алгебры, так как они появились при изучение того, как мы мыслим, какие пути мышления верные, что является доказательством, и других схожих вещей. Булева алгебра названа так в честь английского математика Джорджа Буля, который в 1854 году опубликовал классическую книгу ``An Investigation of the Laws of Thought, on Which Are Founded the Mathematical Theories of Logic and Probabilities.'', в которой Буль намеревался дать математический анализ логики. \\

Булева алгебра была впервые придумана для решения задач, которые возникли в процессе разработки релейных коммутационных схем в 1938 году Клодом Шенноном. В своей статье ``A Symbolic Analysis of Relay and Switching Circuits'' он представил способ представления любых схем, составленных из переключателей и реле, в виде набора математических выражений. Вычисления данных выражений были основаны на булевой алгебре. \\

%========================================
\mysubsection{Мотивация}
Рассмотрим задачу сложения двух бинарных чисел $A$ и $B$. Результат сложения будем представлять в виде двух битов: $S$ – бит суммы и $C$ – бит переноса. Есть всего 4 возможных комбинации двух бинарных входов, запишем полученные результаты: \\

\begin{center}
\begin{tabular}{c c c c c}
\hline
A & B & & C & S \\
\hline \hline
0 & 0 & & 0 & 0 \\
0 & 1 & & 0 & 1 \\
1 & 0 & & 0 & 1 \\
1 & 1 & & 1 & 0 \\
\end{tabular}
\end{center}
\

Если задуматься, то можно обнаружить следующие связи: \\

\begin{itemize}
  \item Бит $C$ равен 1 только в случае, когда оба бита $A$ и $B$ равны 1, иначе он равен 0.
  \item Бит $S$ равен 1, если один из битов $A$ и $B$ равен 1, не не сразу оба.
\end{itemize}
\

Эффективное представление, упрощение и манипуляции над подобными логическими выражениями являются главным объектом изучения в булевой алгебре.

%==================================================
\mysection{Определение и основные тождества}

%========================================
\mysubsection{Определение}
Сформулируем теперь математическое определение булевой алгебры: \\

\mydefinition{
Булевой алгеброй называется непустое множество $A$ с двумя бинарными операциями $\land$ (аналог конъюнкции), $\lor$ (аналог дизъюнкции), одной унарной операцией $\neg$ (аналог отрицания) и двумя выделенными элементами: 0 (или Ложь) и 1 (или Истина) такими, что для любых $a, b, c$ из множества $A$ верны следующие аксиомы:

\begin{center}
\addtolength{\tabcolsep}{12pt}
\begin{tabular}{c c}

\textbf{ассоциативность}: & \\[4pt]
$a \lor (b \lor c) = (a \lor b) \lor c$ & $a \land (b \land c) = (a \land b) \land c$ \\[8pt]
\textbf{коммутативность}: & \\[4pt]
$a \lor b = b \lor a$ & $a \land b = b \land a$ \\[8pt]
\textbf{законы поглощения}: & \\[4pt]
$a \lor (a \land b) = a$ & $a \land (a \lor b) = a$ \\[8pt]
\textbf{дистрибутивность}: & \\[4pt]
$a \lor (b \land c) = (a \lor b) \land (a \lor c)$ & $a \land (b \lor c) = (a \land b) \lor (a \land c)$ \\[8pt]
\textbf{комплементность}: & \\[4pt]
$a \lor \neg a = 1$ & $a \land \neg a = 0$ \\[8pt]

\end{tabular}
\addtolength{\tabcolsep}{-12pt}
\end{center}
}
\

%========================================
\mysubsection{Основные тождества}
Из аксиом видно, что наименьшим элементом является 0, наибольшим является 1, а дополнение $\neg a$ любого элемента $a$ однозначно определено. Для всех $a$ и $b$ из $A$ верны также следующие равенства: \\

\begin{center}
\addtolength{\tabcolsep}{12pt}
\begin{tabular}{c c}

\textbf{законы де Моргана}: & \\[4pt]
$\neg (a \lor b) = \neg a \land \neg b$ & $\neg (a \land b) = \neg a \lor \neg b$ \\[8pt]
\textbf{Блейка-Порецкого}: & \\[4pt]
$a \lor (\neg a \land b) = a \lor b$ & $a \land (\neg a \lor b) = a \land b$ \\[8pt]
\textbf{идемпотентность}: & \\[4pt]
$a \lor a = a$ & $a \land a = a$ \\[8pt]
\textbf{инволютивность}: & \\[4pt]
$\neg \neg a = a$ & \\[8pt]
\textbf{свойства констант}: & \\[4pt]
$a \lor 0 = a$ & $a \land 1 = a$ \\[8pt]
$a \lor 1 = 1$ & $a \land 0 = 0$ \\[8pt]
$\neg 0 = 1$ & $\neg 1 = 0$ \\[8pt]
\textbf{склеивание}: & \\[4pt]
$(a \lor b) \land (\neg a \lor b) = b$ & $(a \land b) \lor (\neg a \land b) = b$ \\[8pt]

\end{tabular}
\addtolength{\tabcolsep}{-12pt}
\end{center}
\

%==================================================
\mysection{Нормальная форма}
Формула в булевой логике может быть записана в дизъюнктивной и в конъюнктивной нормальной форме. \\

\mydefinition{
Дизъюнктивная нормальная форма (ДНФ) в булевой логике — нормальная форма, в которой булева формула имеет вид дизъюнкции конъюнкций литералов.
}
\

\mydefinition{
Конъюнктивная нормальная форма (КНФ) в булевой логике — нормальная форма, в которой булева формула имеет вид конъюнкции дизъюнкций литералов.
}
\

Любая булева формула может быть приведена к ДНФ и к КНФ, следуя простым алгоритмам: \\

\textbf{Алгоритм построения ДНФ / КНФ}
\begin{enumerate}
  \item Избавиться от всех логических операций, содержащихся в формуле, заменив их основными: конъюнкцией, дизъюнкцией, отрицанием.
  \item Заменить знак отрицания, относящийся ко всему выражению, знаками отрицания, относящимися к отдельным переменным высказываниям.
  \item Избавиться от знаков двойного отрицания.
  \item Применить, если нужно, к операциям конъюнкции и дизъюнкции свойства дистрибутивности и формулы поглощения.
\end{enumerate}
\

Для каждой булевой формулы существует бесконечное множество соответствующих ей ДНФ и КНФ. Введём определение уникальной формы: \\

\mydefinition{
Совершенная ДНФ (КНФ) – это такая ДНФ (КНФ), которая удовлетворяет трем условиям:
\begin{itemize}
  \item в ней нет одинаковых элементарных конъюнкций (дизъюнкций).
  \item в каждой конъюнкции (дизъюнкции) нет одинаковых пропозициональных переменных.
  \item каждая элементарная конъюнкция (дизъюнкция) содержит каждый литерал из входящих в данную ДНФ (КНФ) литералов.
\end{itemize}
}
\

Рассмотрим пример получения СДНФ: \\
\myequation{
X \lor \neg Y \land \neg Z
&= X \land (Y \lor \neg Y) \land (Z \lor \neg Z) \lor (X \lor \neg X) \land \neg Y \land \neg Z \\
&= (X \land Y \land Z) \lor (X \land \neg Y \land Z) \lor (X \land Y \land \neg Z) \lor \\
&\lor (X \land \neg Y \land \neg Z) \lor (X \land \neg Y \land \neg Z) \lor (\neg X \land \neg Y \land \neg Z) \\
&= (X \land Y \land Z) \lor (X \land \neg Y \land Z) \lor (X \land Y \land \neg Z) \lor \\
&\lor (X \land \neg Y \land \neg Z) \lor (\neg X \land \neg Y \land \neg Z)
}

\end{document}
