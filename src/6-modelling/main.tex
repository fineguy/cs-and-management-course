\documentclass[14pt]{extarticle} %Класс позволяет использовать базовые шрифты бОльших размеров
\usepackage[utf8x]{inputenc} %кодировка файла макета utf8
\usepackage[russian]{babel} 
\usepackage[left=25mm,right=15mm,top=20mm,bottom=20mm]{geometry} %Попытка разобраться с полями страниц
\usepackage{ntheorem} %окружение для настройки теорем 
\usepackage{graphicx} %работа с рисунками
\usepackage[labelsep=period,figurewithin=none,tablewithin=none]{caption} %подписи к объектам (рисунки, таблицы)
\usepackage{listings} %работа с листингами
\usepackage{indentfirst} %отступ первого абзаца в разделе
\usepackage{enumitem} %настройка маркированных и нумерованных списков (см. примеры настройки в тексте)
\usepackage{url} %формирование ссылок на электронные источники
\usepackage{fancyhdr} %Настройка нумерации страниц
\usepackage{tocloft} %Настройка заголовка для содержания

%====================================================================
% Мои настройки
\usepackage{amsmath}
\usepackage{amssymb}
\usepackage{booktabs}
\usepackage{caption}
\usepackage{subfig}
\usepackage{algorithm}
\usepackage{algorithmic}
\usepackage{bm}
\usepackage[para,online,flushleft]{threeparttable}
%\graphicspath{ {imgs/} }
\usepackage{xifthen}

%====================================================================

%====================================================================
%Настройки макета
%----------------
%Содержимое этого блока не должно подвергаться изменению
%====================================================================
\selectlanguage{russian}
\setlength{\parindent}{1.27cm}

%---------Размеры страницы
%\textwidth=11.3cm \textheight=17cm 
%\hoffset=-30mm \voffset=-15mm

%---------Настройка подписей к таблицам
\DeclareCaptionFormat{mplain}{#1#2\par \centering #3\par}
\captionsetup[table]{format=mplain,
justification=raggedleft,%
labelsep=none,%
singlelinecheck=false,%
skip=3pt}

%---------Настройка подписей к таблицам
%\setbeamertemplate{caption}[numbered]
%\setbeamertemplate{footline}[frame number]
\newcommand{\rulesep}{\unskip\ \vrule\ }
\newcommand{\source}[1]{\caption*{\textbf{Источник}: {#1}} }

%Настройка нумерации страниц
\fancyhf{} % clear all header and footers
\renewcommand{\headrulewidth}{0pt} % remove the header rule
\rfoot{\small \thepage}
\pagestyle{fancy}

%Настройка заголовка для содержания
\renewcommand{\cfttoctitlefont}{\hfill\normalfont\large\bfseries}
\renewcommand{\cftaftertoctitle}{\hfill\thispagestyle{empty}} 

%Настройка теорем
\theoremseparator{.}

%---------Команды рубрикации--------------

%Заголовки
\makeatletter
\renewcommand{\section}{\@startsection{section}{1}%
{\parindent}{-3.5ex plus -1ex minus -.2ex}%
{2.3ex plus.2ex}{\normalfont\large\bfseries}}

\renewcommand{\subsection}{\@startsection{subsection}{2}%
{\parindent}{-3.5ex plus -1ex minus -.2ex}%
{1.5ex plus.2ex}{\normalfont\large\bfseries}}

\renewcommand{\subsubsection}{\@startsection{subsubsection}{3}%
{\parindent}{-1.5ex plus -1ex minus -.2ex}%
{0.5ex plus.2ex}{\normalfont\bfseries}}
\makeatother

%Команда уровня главы
\newcommand{\mysection}[1]{
 \newpage
 \refstepcounter{section}
 {
  \section*{Глава \thesection. #1 \raggedright }
 }
 \addcontentsline{toc}{section}{Глава \thesection. #1} 
}

%Команда уровня параграфа
\newcommand{\mysubsection}[1]{
 \refstepcounter{subsection}
 \subsection*{\thesubsection. #1}
 \addcontentsline{toc}{subsection}{\thesubsection. #1}
}

%Команда третьего уровня
\newcommand{\mysubsubsection}[1]{
\refstepcounter{subsubsection}
% \addcontentsline{toc}{subsubsection}{\thesubsubsection. #1}
\subsubsection*{#1}
}

%Оформление Приложений
\newcounter{appendix}
\newcommand{\addappendix}[1]{
 \newpage
 \refstepcounter{appendix} 
 \section*{ПРИЛОЖЕНИЕ \theappendix. \\#1}
 \addcontentsline{toc}{section}{ПРИЛОЖЕНИЕ \theappendix. #1}
}

%Команда ненумерованной главы
\newcommand{\mynonumbersection}[1]{
\newpage
{
%	\begin{center}\section*{#1}\end{center}
	\centering\section*{#1}
}
\addcontentsline{toc}{section}{#1} 
}

%--------Настройка маркированных и нумерованных списков
\setlist{itemsep=0pt,topsep=0pt}

%--------Настройка листингов программного кода
\lstloadlanguages{C,[ANSI]C++}%!настройка листинга
%Можно подключить другие языки (см документацию к пакету)

%--------Тонкая настройка листингов
\lstset{
inputencoding=utf8x,
extendedchars=false,
showstringspaces=false,
showspaces=false,
keepspaces = true,
basicstyle=\small\ttfamily,
keywordstyle=\bfseries,
tabsize=2,                      % sets default tabsize to 2 spaces
captionpos=t,                   % sets the caption-position to bottom
breaklines=true,                % sets automatic line breaking
breakatwhitespace=true,        % sets if automatic breaks should only happen at whitespace
title=\lstname,                 % show the filename of files included with \lstinputlisting;
basewidth={0.5em,0.45em},
}

%----------Настройка подписей к листингам
\renewcommand{\lstlistingname}{Листинг}

%------------Подключение стиля для оформления списка литературы
\makeatletter
\renewcommand{\@biblabel}[1]{#1.\hfill}
\makeatother
\bibliographystyle{ugost2003s}
\PrerenderUnicode{ЙЦУКЕНГШЩЗХЪЭЖДЛОРПАВЫФЯЧСМИТЬБЮйцукенгшщзхъэждлорпавыфячсмитьбю}


%----------Use frames for text
\usepackage{mdframed}
\newcommand{\myframe}[1]{
\begin{mdframed}
\begin{center}
#1
\end{center}
\end{mdframed}
}

%----------Setup images
\captionsetup{figurename=Рисунок}

\newcommand{\myimage}[4]{ % source?; width; caption; img
\begin{figure}[h]
\centering
\includegraphics[width=#2\textwidth]{#4}
\caption{#3}
\ifthenelse{\equal{#1}{}}{
}{
\source{#1}
}
\end{figure}
}

%----------Setup Math stuff
\newtheorem{definition}{Определение}[section]
\newtheorem{proof}{Доказательство}
\newtheorem{theorem}{Теорема}

\makeatletter
\renewcommand*{\ALG@name}{Алгоритм}
\makeatother

\newcommand{\mydefinition}[1]{
\myframe{
\theoremstyle{definition}
\begin{definition}
#1
\end{definition}
}}

\newcommand{\myproof}[1]{
\theoremstyle{proof}
\begin{proof}
#1
\end{proof}
}

\newcommand{\mytheorem}[1]{
\myframe{
\theoremstyle{theorem}
\begin{theorem}
#1
\end{theorem}
}}

\newcommand{\myequation}[2][]{
\begin{equation}
\ifthenelse{\equal{#1}{}}{
\begin{aligned}#2,\end{aligned}
}{
\left.\begin{aligned}#2\end{aligned}\right\} #1
}
\end{equation}
}

\graphicspath{ {imgs/} }

\begin{document}

%==================================================
\mynonumbersection{МАТЕМАТИЧЕСКОЕ И ИМИТАЦИОННОЕ МОДЕЛИРОВАНИЕ}

%==================================================
\mysection{Введение}
Модели описывают наши представления о том, как устроен мир. В математическом моделировании мы переводим эти представления на язык математики. Это даёт много преимуществ: \\

\begin{enumerate}
  \item Математика это очень точный язык. Это помогает нам формулировать идеи и определять основные допущения.
  \item Математика это лаконичный язык с хорошо определенными правилами для работы с ним.
  \item В нашем распоряжении все результаты, которые были получены математиками в течение сотен лет.
  \item Могут быть использованы компьютеры для проведения численных расчетов.
\end{enumerate}
\ 

Математическое моделирование часто приводит к некоторым компромиссам. В реальном мире большая часть взаимодействующих систем слишком сложные для их полного моделирования. Таким образом первый уровень компромиссов заключается в том, чтобы определить самые важные части системы. Только они будут учитываться в модели, всё остальное будет проигнорировано. Второй уровень компромиссов заключается в том, насколько много математических манипуляция мы можем себе позволить. Хотя математика имеет возможность получать общие результаты, эти результаты серьезно зависят от используемых формул. Небольшие изменения в этих формулах могут требовать огромных изменений в математических методах. Использование компьютеров для обработки уравнений в моделях может никогда не приводить к ``красивым'' результатам, но этот подход гораздо более устойчив к изменениям формул. \\

%========================================
\mysubsection{Цели математического моделирования}
Математическое моделирование может быть использовано по огромному количеству причин. Насколько хорошо та или иная цель будет достигнута зависит от уровня знаний о системе и того, насколько хорошо проведено моделирование. Несколько примеров таких целей: \\

\begin{enumerate}
  \item Развитие научного понимания через численное выражение текущего понимания системы. Это покажет как то, что мы уже знаем, так и то, что мы еще не знаем.
  \item Тестирование эффекта от изменений в системе.
  \item Помощь в принятии решений.
\end{enumerate}
\

%========================================
\mysubsection{Классификация математических моделей}
Один из методов классификаций моделей основывается на типе даваемых предсказаний. Детерминированные модели игнорируют случайные изменения, и тем самым всегда выдаёт одни и те же предсказания при одних и тех же входных данных. С другой стороны модели могут быть более статистическими по своей природе и могут предсказывать распределение возможных исходов. Такие модели называются стохастическими. \\

Другой метод классификации моделей различает их по уровню понимания, на котором основываются модели. Самое простое объяснение это учёт иерархии в организационных структурах внутри моделируемой системы. Для животных, например, возможна такая иерархия: \\

\myimage{}{0.4}{}{animals}
\

Модели, которые используют большое количество теоретической информации, общим образом описывают что происходит на одном уровне иерархии, путем рассмотрения процессов на более низких уровнях, называются механистическими моделями, потому что они учитывают механизмы, через которые происходят эти изменения. В эмпирических моделях этих механизмы не учитываются. Вместо этого отмечается, что они происходят, и модель пытается численно учесть изменения соответствующие разным условиям. \\

Два типа классификации: детерминированные / стохастические и механистические / эмпирические, представляют крайности диапазона всех моделей. В промежутке лежит целый спектр моделей. Также эти два типа классификации являются дополняющими друг друга. Возможные пример моделей из этих категорий: \\

\begin{center}
\begin{tabular}{p{0.3\textwidth} | p{0.3\textwidth} p{0.3\textwidth} }
& Эмпирические & Механистические \\
\hline \\
Детерминированные & Предсказание роста численности скота через регрессионную зависимость от потребления корма & Движение планет, основанное на Ньютоновской механике (дифференциальные уравнения) \\
& & \\
Стохастические & Анализ вариации различных урожаев на разных полях и в разные годы & Генетическое многообразие в маленьких популяциях, основанное на законах Менделя (вероятностные уравнения) \\
\end{tabular}
\end{center}
\

%========================================
\mysubsection{Стадии моделирования}
Полезно разделять процесс моделирования на 4 широкие категории действий: построение, изучение, тестирование и использование. Хотелось бы думать, что процесс моделирования проходит плавно от построения до использования, но это редкий случай. Чаще проблемы, найденные в процессе изучения и тестирования, исправляются и возвращаются на шаг построения. Заметим, что если в этот момент в модель вносятся изменения, то шаги изучения и тестирования должны быть повторены. \\

\myimage{}{0.4}{Стадии моделирования}{stages}
\

Такой процесс повторяющихся итераций типичен для проектов по моделированию и является одним из наиболее полезных аспектов моделирования в том смысле, что позволяет нам улучшать понимание того, как работает система. \\

%==================================================
\mysection{Построение моделей}
Перед тем, как начать проект по моделированию, нам нужно четко понять наши цели. Они определяют будущее направление проекта двумя образами. \\

Во-первых, уровень детальности модели зависит от цели, для которой эта модель будет использована. Например, при моделировании роста животного для помощи сельскохозяйственному бизнесу эмпирическая модель, включающая в себя наиболее важные показатели, определяющие рост животного, может быть вполне подходящей. Модель может восприниматься как суммаризация текущего понимания. Такая модель по естественным причинам слабо подходит для исследования процесса вскармливания. \\

Во-вторых, мы должны сделать различие между системой, которую моделируем, и её окружением. Это легко сделать в случае, когда окружение влияет на поведение системы, но сама система не оказывает влияния на окружение. Например, при моделировании роста хвойной плантации для предсказания урожая древесины стоит учитывать погоду как часть окружающей среды. Её влияние на рост может быть учтено через учёт статистики по климату в похожих локациях в близкие годы. Однако любая модель роста мировых лесов почти наверняка должна включать в себя показатели влияния роста на погоду. Лесное покрытие имеет сильное влияние на погоду из-за изменений в уровне углекислого газа в атмосфере. \\

%========================================
\mysubsection{Системный анализ}

%==============================
\mysubsubsection{Предположения}
Определив систему для моделирования, нам необходимо построить базовую структуру модели. Она будет отражать нашу веру о том, как устроена система. Эти убеждения могут быть выражены в виде основополагающих допущений. Будущий анализ системы исходит из того, что эти убеждения верные, но результаты такого анализа верен настолько, насколько верны эти предположения. \\

%==============================
\mysubsubsection{Структурная схема}
Там, где моделируемая система более сложная, мы не можем просто перейти от предположений к уравнениям. Мы должны быть более методичными в случае описания системы и, когда вводятся предположения. Структурные схемы это визуальная помощь к этим задачам. В своей самой простой форме они состоят из последовательности боксов, соединенных сетью из стрелок. Боксы представляют физические сущности, которые есть в системе, а стрелки представляют способ, которым эти сущности связаны друг с другом. \\

%========================================
\mysubsection{Выбор математических уравнений}
После того, как устройство системы было определено, должны быть выбраны математические уравнения для описания системы. Стоит осторожно выбирать эти уравнения – они могут иметь непредвиденные эффекты на поведение модели. \\

%==============================
\mysubsubsection{Уравнения из литературы}
Вполне возможно, что кто-то уже опубликовал уравнения относящиеся к величинам, в которых мы заинтересованы. Это даёт хорошую начальную точку, но продвигаться нужно с осторожностью. Проблемы, с которыми можно столкнуться: \\

\begin{itemize}
  \item уравнения выведены из данных в интервале, который не покрывает интервал, требуемый для применения модели.
  \item экспериментальные условия (окружение) значительно отличаются от условий, встречающихся при применении модели.
  \item уравнения описывают поведение в среднем без учета отклонений на границах.
\end{itemize}
\

Некоторые области науки настолько хорошо изучены, что подходящие формы анализа стали стандартом. Исходя из этого относительно безопасно предположить, что похожий анализ (и тем самым устройство уравнений) переносится на похожие проблемы. \\

%==============================
\mysubsubsection{Аналогии из физики}
Физики построили математические модели для описания широкого спектра систем. Часто системы могут указаны точно, делая применения математических формул относительно простым действием. \\

%==============================
\mysubsubsection{Исследование данных}
Когда нет никакой информации о форме соотношений, единственным способом остается получение данных и подгон уравнений под них. Это даёт преимущество в том смысле, что мы имеем контроль на анализом. \\


%========================================
\mysubsection{Решение уравнений}

%==============================
\mysubsubsection{Аналитически}
Аналитическое решение модели может дать многое. Это позволит нам производить все манипуляции, предполагаемые моделью, с минимальными усилиями. Заметим, что полное аналитическое решение для стохастической модели включает в себя нахождение распределения исходов, но нам может хватить и уравнений, описывающих среднее и отклонение. В общем случае получение аналитического решения почти никогда не является простой задачей. \\

%==============================
\mysubsubsection{Числено}
Когда аналитические методы непродуктивны, мы можем использовать численные методы для получения приближенных решений. Хотя они никогда не будут иметь тот же уровень общности, что и аналитические решения, но они могут быть также пригодны для практического использования. \\

Численное решение уравнений модели обычно имитирует процесс, описываемый моделью. Для дифференциальных решений численное решение является точным, так как мы можем использовать правила заложенные в этих уравнениях для отслеживания эволюции системы. В случае стохастической модели мы можем повторно симулировать исходы, используя генератор случайных чисел, и объединить большое количество симуляций для приближения распределения исходов. 

\end{document}
