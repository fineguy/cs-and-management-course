\documentclass[14pt]{extarticle} %Класс позволяет использовать базовые шрифты бОльших размеров
\usepackage[utf8x]{inputenc} %кодировка файла макета utf8
\usepackage[russian]{babel} 
\usepackage[left=25mm,right=15mm,top=20mm,bottom=20mm]{geometry} %Попытка разобраться с полями страниц
\usepackage{ntheorem} %окружение для настройки теорем 
\usepackage{graphicx} %работа с рисунками
\usepackage[labelsep=period,figurewithin=none,tablewithin=none]{caption} %подписи к объектам (рисунки, таблицы)
\usepackage{listings} %работа с листингами
\usepackage{indentfirst} %отступ первого абзаца в разделе
\usepackage{enumitem} %настройка маркированных и нумерованных списков (см. примеры настройки в тексте)
\usepackage{url} %формирование ссылок на электронные источники
\usepackage{fancyhdr} %Настройка нумерации страниц
\usepackage{tocloft} %Настройка заголовка для содержания

%====================================================================
% Мои настройки
\usepackage{amsmath}
\usepackage{amssymb}
\usepackage{booktabs}
\usepackage{caption}
\usepackage{subfig}
\usepackage{algorithm}
\usepackage{algorithmic}
\usepackage{bm}
\usepackage[para,online,flushleft]{threeparttable}
%\graphicspath{ {imgs/} }
\usepackage{xifthen}

%====================================================================

%====================================================================
%Настройки макета
%----------------
%Содержимое этого блока не должно подвергаться изменению
%====================================================================
\selectlanguage{russian}
\setlength{\parindent}{1.27cm}

%---------Размеры страницы
%\textwidth=11.3cm \textheight=17cm 
%\hoffset=-30mm \voffset=-15mm

%---------Настройка подписей к таблицам
\DeclareCaptionFormat{mplain}{#1#2\par \centering #3\par}
\captionsetup[table]{format=mplain,
justification=raggedleft,%
labelsep=none,%
singlelinecheck=false,%
skip=3pt}

%---------Настройка подписей к таблицам
%\setbeamertemplate{caption}[numbered]
%\setbeamertemplate{footline}[frame number]
\newcommand{\rulesep}{\unskip\ \vrule\ }
\newcommand{\source}[1]{\caption*{\textbf{Источник}: {#1}} }

%Настройка нумерации страниц
\fancyhf{} % clear all header and footers
\renewcommand{\headrulewidth}{0pt} % remove the header rule
\rfoot{\small \thepage}
\pagestyle{fancy}

%Настройка заголовка для содержания
\renewcommand{\cfttoctitlefont}{\hfill\normalfont\large\bfseries}
\renewcommand{\cftaftertoctitle}{\hfill\thispagestyle{empty}} 

%Настройка теорем
\theoremseparator{.}

%---------Команды рубрикации--------------

%Заголовки
\makeatletter
\renewcommand{\section}{\@startsection{section}{1}%
{\parindent}{-3.5ex plus -1ex minus -.2ex}%
{2.3ex plus.2ex}{\normalfont\large\bfseries}}

\renewcommand{\subsection}{\@startsection{subsection}{2}%
{\parindent}{-3.5ex plus -1ex minus -.2ex}%
{1.5ex plus.2ex}{\normalfont\large\bfseries}}

\renewcommand{\subsubsection}{\@startsection{subsubsection}{3}%
{\parindent}{-1.5ex plus -1ex minus -.2ex}%
{0.5ex plus.2ex}{\normalfont\bfseries}}
\makeatother

%Команда уровня главы
\newcommand{\mysection}[1]{
 \newpage
 \refstepcounter{section}
 {
  \section*{Глава \thesection. #1 \raggedright }
 }
 \addcontentsline{toc}{section}{Глава \thesection. #1} 
}

%Команда уровня параграфа
\newcommand{\mysubsection}[1]{
 \refstepcounter{subsection}
 \subsection*{\thesubsection. #1}
 \addcontentsline{toc}{subsection}{\thesubsection. #1}
}

%Команда третьего уровня
\newcommand{\mysubsubsection}[1]{
\refstepcounter{subsubsection}
% \addcontentsline{toc}{subsubsection}{\thesubsubsection. #1}
\subsubsection*{#1}
}

%Оформление Приложений
\newcounter{appendix}
\newcommand{\addappendix}[1]{
 \newpage
 \refstepcounter{appendix} 
 \section*{ПРИЛОЖЕНИЕ \theappendix. \\#1}
 \addcontentsline{toc}{section}{ПРИЛОЖЕНИЕ \theappendix. #1}
}

%Команда ненумерованной главы
\newcommand{\mynonumbersection}[1]{
\newpage
{
%	\begin{center}\section*{#1}\end{center}
	\centering\section*{#1}
}
\addcontentsline{toc}{section}{#1} 
}

%--------Настройка маркированных и нумерованных списков
\setlist{itemsep=0pt,topsep=0pt}

%--------Настройка листингов программного кода
\lstloadlanguages{C,[ANSI]C++}%!настройка листинга
%Можно подключить другие языки (см документацию к пакету)

%--------Тонкая настройка листингов
\lstset{
inputencoding=utf8x,
extendedchars=false,
showstringspaces=false,
showspaces=false,
keepspaces = true,
basicstyle=\small\ttfamily,
keywordstyle=\bfseries,
tabsize=2,                      % sets default tabsize to 2 spaces
captionpos=t,                   % sets the caption-position to bottom
breaklines=true,                % sets automatic line breaking
breakatwhitespace=true,        % sets if automatic breaks should only happen at whitespace
title=\lstname,                 % show the filename of files included with \lstinputlisting;
basewidth={0.5em,0.45em},
}

%----------Настройка подписей к листингам
\renewcommand{\lstlistingname}{Листинг}

%------------Подключение стиля для оформления списка литературы
\makeatletter
\renewcommand{\@biblabel}[1]{#1.\hfill}
\makeatother
\bibliographystyle{ugost2003s}
\PrerenderUnicode{ЙЦУКЕНГШЩЗХЪЭЖДЛОРПАВЫФЯЧСМИТЬБЮйцукенгшщзхъэждлорпавыфячсмитьбю}


%----------Use frames for text
\usepackage{mdframed}
\newcommand{\myframe}[1]{
\begin{mdframed}
\begin{center}
#1
\end{center}
\end{mdframed}
}

%----------Setup images
\captionsetup{figurename=Рисунок}

\newcommand{\myimage}[4]{ % source?; width; caption; img
\begin{figure}[h]
\centering
\includegraphics[width=#2\textwidth]{#4}
\caption{#3}
\ifthenelse{\equal{#1}{}}{
}{
\source{#1}
}
\end{figure}
}

%----------Setup Math stuff
\newtheorem{definition}{Определение}[section]
\newtheorem{proof}{Доказательство}
\newtheorem{theorem}{Теорема}

\makeatletter
\renewcommand*{\ALG@name}{Алгоритм}
\makeatother

\newcommand{\mydefinition}[1]{
\myframe{
\theoremstyle{definition}
\begin{definition}
#1
\end{definition}
}}

\newcommand{\myproof}[1]{
\theoremstyle{proof}
\begin{proof}
#1
\end{proof}
}

\newcommand{\mytheorem}[1]{
\myframe{
\theoremstyle{theorem}
\begin{theorem}
#1
\end{theorem}
}}

\newcommand{\myequation}[2][]{
\begin{equation}
\ifthenelse{\equal{#1}{}}{
\begin{aligned}#2,\end{aligned}
}{
\left.\begin{aligned}#2\end{aligned}\right\} #1
}
\end{equation}
}



\begin{document}

%==================================================
\mysection{Мотивация}

Начнём с вопроса, что такое информация? Давайте рассмотрим пример: \\

\begin{itemize}
  \item Вопрос: ``Температура в Москве сейчас выше 15 градусов?'' \\
           На него возможны ответы либо ``да'' , либо ``нет''.
  \item Вопрос: ``Президент Российской федерации поговорил с определенным человеком в Москве. С кем?'' \\
  	   На него возможно ответить более чем 10 миллионами способов.
\end{itemize}
\

Очевидно, что второй вопрос даёт нам гораздо больше информации, чем первый.

\myframe{Количество возможных ответов связано с ``информацией''.}

Посмотрим на другой пример: \\

\begin{itemize}
  \item Вы бросаете игральную кость один раз. Есть 6 возможных исходов. Вы записываете полученный исход и говорите о нём другу. Таким образом вы передали своему другу определенный объем информации.
  \item Вы бросаете игральную кость три раза. Опять же вы записываете все полученные исходы и рассказываете о них другу. Очевидно, что в этом случае вы передали своему другу в три раза больше информации.
\end{itemize}

\myframe{``Информация'' должна обладать аддитивностью.}

Заметим, что во второй ситуации возможны $6^3$ исходов, что в 36 раз больше, чем в первой ситуации. Но количество информации возросло всего в три раза. Как с этим быть? Довольно логичным кажется использовать логарифм от количества исходов для того, чтобы измерить количество информации. Именно это в 1928 году предложил американский ученый-электронщик Ральф Хартли: \\

\mydefinition{
Мы определим меру информации:
\myequation{ I(U) \triangleq \log_b{r} }
где $r$ это количество всевозможных исходов для случайного сообщения $U$.
}

Используя это определение, легко убедиться, что оно удовлетворяет свойству аддитивности: 

\myequation{ I(U_1, U_2, \ldots, U_n) = \log_b{r^n} = n \cdot \log_b{r} = n I(U_1) }

Хартли также корректно отмечал, что основание логарифма $b$ не имеет большого значения. Оно лишь определяет какие единицы измерения используются. Для некоторых особых значения $b$ есть собственные названия такие единиц измерения: \\

\begin{itemize}
  \item $b = 2 \, (\log_2)$ – бит;
  \item $b = e \, (\ln)$ – нат (натуральный логарифм);
  \item $b = 10 \, (\log_{10})$ – Хартли.
\end{itemize}
\

С определением данным Хартли есть фундаментальная проблема – согласно нему минимальное ненулевое количество информации это $\log_2{2} = 1$ бит. Может показаться, что это небольшой объем информации, но представим, что мы хотим записать номера всех $7`621`000`000$ людей на планете. Согласно данному определению, нам понадобится $\log_2{(7`621`000`000)} \approx 32.8$ битов. То есть, имея информации всего в 33 раза больше, чем 1 бит, можно раздать каждому человеку уникальный телефонный номер. \\

Чтобы еще лучше разобраться, в чем проблема, представим что у нас есть два мешка: в первом лежит 2 черных шара и 2 белых; а во втором лежат 3 черных шара и 1 белый. Давайте случайно вытаскивать шар из мешка, и пусть $U$ будет цветом вынутого шара. В каждом мешке есть шары двух цветов, таким образом $I(U_A) = I(U_B) = \log_2{2} = 1$ бит. Но очевидно, что вытаскивая из второго мешка черный шар, мы получаем меньше информации, так как мы изначально ожидаем такого исхода. 

\myframe{Хорошая мера ``информации'' должна учитывать вероятности возможных исходов.}

Впервые к такому выводу пришел американский математик Клод Элвуд Шеннон в 1948 году в статье ``A Mathematical Theory of Communication''. 

\mydefinition{
Шенноновская мера информация является ``усредненной информацией Хартли'':
\myequation{ \sum_{i=1}^r p_i \log_2 \frac{1}{p_i} = - \sum_{i=1}^r p_i \log_2 p_i }
где $p_i$ обозначает вероятность $i$-го возможного исхода.
}

%==================================================
\mysection{Энтропия}

%========================================
\mysubsection{Определение}

Теперь мы формально определим Шенноновскую меру ``информации''. В силу связи с похожими концептами в разных разделах физики, Шеннон назвал эту меру \textit{энтропией}. 

\mydefinition{
Энтропия дискретной величины $U$, которая принимает значения из множества $\mathcal{U}$ (алфавит) определяется как:
\myequation{ H(U) \triangleq - \sum_{u \in supp(P_U)} P_U(u) \log_b P_U(u) }
где $P_U(\cdot)$ обозначает функцию вероятности случайной величины $U$, и где носитель $P_U$ определен как:
\myequation{ supp(P_U) \triangleq \{ u \in \mathcal{U}: P_U(u) > 0 \} }
Другая часто используемая форма записи:
\myequation{ H(U) = E_U [ -\log_b P_U(U) ] }
}

Заметим, что $lim_{t \rightarrow 0} t \log_b t = 0$, поэтому во многих случаях мы не будем упоминать носитель при суммировании по $P_U(u)$, подразумевая, что мы исключили все $u$ с нулевой вероятностью. \\

Также важно отметить, что энтропия случайной величины $U$ никак не зависит от возможных значений $U$, а только зависит от вероятностей этих значений.

%\mydefinition{
%Энтропия случайного дискретного вектора $W = (X, Y)^T$ определяется как:
%\myequation{
%H(W) = H(X, Y) &\triangleq E_{X, Y} [ -\log_b P_{X, Y} (X, Y) ] \\
%&= -\sum_{(x, y) \in supp(P_{X, Y})} P_{X, Y}(x, y) \log_b P_{X, Y}(x, y) }
%где $P_{X, Y}(\cdot, \cdot)$ обозначает функцию совместной вероятности для $(X, Y)$.
%}

\mydefinition{
Условная энтропия случайной величины $X$ при условии события $Y=y$ определяется как:
\myequation{
H(X|Y=y) &\triangleq -\sum_{x \in supp(P_{X|Y} (\cdot|y))} P_{X|Y}(x|y) \log P_{X|Y}(x|y) \\
&= E [ -\log P_{X|Y}(X|Y) | Y=y ] }
Заметим, что определение идентично предыдущему с единственно разницей, что всё обусловлено на событие $Y=y$.
}

\mydefinition{
Условная энтропия случайной величины $X$ при условии случайной величины $Y$ определяется как:
\myequation{
H(X|Y) &\triangleq -\sum_{y \in supp(P_Y)} P_Y(y) \cdot H(X|Y=y) \\
&= E_Y [ H(X|Y=y) ] \\
&= -\sum_{(x,y) \in supp(P_{X,Y}) } P_{X,Y}(x,y) \log P_{X|Y}(x|y) \\
&= E [ -\log P_{X|Y} (X|Y) ] }
Заметим, что определение идентично предыдущему с единственно разницей, что всё обусловлено на событие $Y=y$.
}

%========================================
\mysubsection{Аксиоматическое определение}

Может показаться странным, почему энтропия имеет именно такое определение. Однако в своей исходной статье Шеннон показал, что данное определение энтропии может быть получено естественным путем, приняв за основу определенную систему аксиом. Обозначим вероятностное распределение над $m$ буквами как $P=(p_1, \ldots, p_m)$ и рассмотрим функционал $H_m (p_1, \ldots, p_m)$. Если $H_m$ удовлетворяет аксиомам: \\

\begin{enumerate}
  \item Инвариантность относительно перестановок.
  \item Раширяемость: $H_m (p_1, \ldots, p_{m-1}, 0) = H_{m-1} (p_1, \ldots, p_{m-1})$.
  \item Нормализация: $H_2(\frac{1}{2}, \frac{1}{2}) = \log 2$.
  \item Субаддитивность: $H(X,Y) \leq H(X) + H(Y)$.
  \item Аддитивность: $H(X,Y) = H(X) + H(Y)$, если $X$ ортоганален $Y$.
  \item Непрерывность: $H_2(p, 1-p) \rightarrow 0$ при $p \rightarrow 0$.
\end{enumerate}
\

тогда $H_m (p_1, \ldots, p_m) = \sum_{i=1}^m p_i \log \frac{1}{p_i}$ единственно возможный вариант.

%========================================
\mysubsection{Cвойства}

Энтропия Шеннона обладает несколькими важными свойствами: \\

\begin{enumerate}
  \item (Неотрицательность). $H(X) \geq 0$, причем равенство возможно тогда и только тогда, когда случайная величина $X$ константна.
  \item (Равномерное распределение максимизирует энтропию). Для конечного множества $\mathcal{X}, H(X) \leq \log | \mathcal{X} |$, причем равенство возможно тогда и только тогда, когда случайная величина $X$ имеет равномерное распределение над $\mathcal{X}$.
  \item (Инвариантность относительно перестановки). $H(X) = H(f(X))$ для любой биекции $f$.
  \item (Малое цепное правило). $H(X,Y) = H(X) + H(Y|X) \leq H(X) + H(Y)$
  \item (Полное цепное правило). $H(X_1, \ldots, X_n) = \sum_{i=1}^n H(X_i | X^{i-1}) \leq \sum_{i=1}^n H(X_i)$
  \item (Обусловленность снижает энтропию). $H(X|Y) \leq H(X)$, причем равенство возможно тогда и только тогда, когда случайные величины $X$ и $Y$ независимы. 
\end{enumerate}
\

Докажем эти свойства: \\

\begin{enumerate}
  \item Матожидание положительной функции также положительно.
  \item Минус логарифм является выпуклой функцией на интервале $(0, 1)$, поэтому это следует из неравенства Йенсена.
  \item $H$ зависит только от значений $P_X$, но не от позиции аргументов.
  \item \myequation{
H(X,Y) &= E [ \log \frac{1}{P_{X,Y} (X,Y)} ] = E [ \log \frac{1}{P_X(X) \cdot P_{Y|X} (Y|X)} ] \\ 
&= E [ \log \frac{1}{P_X(X)} ] + E [ \log \frac{1}{P_{Y|X} (Y|X)} ] \\ 
&= H(X) + H(Y|X) 
}
  \item Отображение $x \mapsto (x, f(x))$ является биекцией, поэтому: \myequation{
H(X) = H(X, f(X)) = H(f(X)) + H(X|f(X)) \geq H(f(X))
}
  \item \myequation{
H(X|Y) - H(X) &= E [ \log \frac{P_X(X)}{P_{X|Y}(X|Y)} ] \\
&= E [ \log \frac{P_X(X) \cdot P_Y(Y)}{P_{X|Y}(X|Y) \cdot P_Y(Y)} ] \\
&= E [ \log \frac{P_X(X) P_Y(Y)}{P_{X,Y}(X,Y)} ] \\
&= \sum_{(x,y) \in supp(P_{X,Y})} P_{X,Y}(x,y) \log \frac{P_X(x) P_Y(y)}{P_{X,Y}(x,y)} \\
&\leq \sum_{(x,y) \in supp(P_{X,Y})} P_{X,Y}(x,y) \Big( \frac{P_X(x) P_Y(y)}{P_{X,Y}(x,y)} - 1 \Big) \cdot \log{e} \\
&= \sum_{(x,y) \in supp(P_{X,Y})} (P_X(x)P_Y(y) - P_{X,Y}(x,y)) \cdot \log{e} \\
&= \Bigg( \sum_{(x,y) \in supp(P_{X,Y})} P_X(x)P_Y(y) - 1 \Bigg) \cdot \log{e} \\
&\leq \Bigg( \sum_{x \in X, y \in Y} P_X(x)P_Y(y) - 1 \Bigg) \cdot \log{e} \\
&= \Bigg( \sum_{x \in X} P_X(x) \sum{y \in Y} P_Y(y) - 1 \Bigg) \cdot \log{e} \\
&= (1-1) \log{e} = 0
}
Важно, что при этом $H(X|Y=y)$ может быть как меньше, так и больше $H(X)$.

\end{enumerate}

%==================================================
\mysection{Совместная информация}

%========================================
\mysubsection{Определение}

Наконец мы добрались до понятия информации. Представьте, что у нас есть случайная величина $X$ с энтропией $H(X)$. Как измерить количество информации, которое даёт другая случайная величина $Y$ об $X$? Логично будет замерить энтропию $X$ до и после того, как мы узнали об $Y$! \\

\mydefinition{
Совместная информация между случайными величинами $X$ и $Y$ определяется как:
\myequation{ I(X; Y) \triangleq H(X) - H(X|Y) }
}

Заметим, что это именно совместная информация, а не информация об $X$, которую даёт $Y$. Это легко увидеть, дважды воспользовавшись цепным правилом:

\myequation{
& H(X,Y) = H(X) + H(Y|X) = H(Y) + H(X|Y) \\
&\Rightarrow H(X) - H(X|Y) = H(Y) - H(Y|X) \\
&\Rightarrow I(X;Y) = I(Y;X)
}

Аналогично тому, как мы определяли условную энтропию, можно определить и условную совместную информацию. Например:

\mydefinition{
\myequation{
I(X;Y|Z) &\triangleq E_Z [ I(X;Y| Z=z) ] \\
&= \sum_z P_Z(z) (H(X|Z=z) - H(X|Y,Z=z)) \\
&= H(X|Z) - H(X|Y,Z)
}}

%========================================
\mysubsection{Cвойства}

Многие свойства совместной информации следуют из свойств энтропии. 

\mytheorem{
Пусть $X$ и $Y$ являются случайными величинами с совместной информацией $I(X;Y)$. Тогда:
\myequation{ 0 \leq I(X;Y) \leq \min \{ H(X), H(Y) \} }
Равенство в левой части достигается тогда и только тогда, когда $P_{X,Y} = P_X \cdot P_Y$. Равенство в правой части достигается тогда и только тогда, когда $X$ определяет $Y$, либо наоборот.
}

\myproof{
Так как обусловленность уменьшает энтропию, то:
\myequation{ I(X;Y) = H(Y) - H(Y|X) \geq H(Y) - H(Y) = 0}
причем равенство возможно только когда $H(Y|X) = H(Y)$. Чтобы доказать правую часть, воспользуемся неотрицательностью энтропии:
\myequation[\Rightarrow I(X;Y) \leq \min \{ H(X), H(Y) \}]{
  I(X;Y) = H(X) - H(X|Y) \leq H(X) \\
  I(X;Y) = H(Y) - H(Y|X) \leq H(Y)
}
причем равенство возможно только когда $H(X|Y)=0$, либо $H(Y|X)=0$, т.е. либо $Y$ задаёт $X$, либо наоборот.
}

Заметим, что совместная информация случайной величины с самой собой будет в точности её энтропия:

\myequation{ I(X; X) = H(X) - H(X|X) = H(X) }

\end{document}

