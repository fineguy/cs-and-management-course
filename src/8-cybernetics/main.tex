\documentclass[14pt]{extarticle} %Класс позволяет использовать базовые шрифты бОльших размеров
\usepackage[utf8x]{inputenc} %кодировка файла макета utf8
\usepackage[russian]{babel} 
\usepackage[left=25mm,right=15mm,top=20mm,bottom=20mm]{geometry} %Попытка разобраться с полями страниц
\usepackage{ntheorem} %окружение для настройки теорем 
\usepackage{graphicx} %работа с рисунками
\usepackage[labelsep=period,figurewithin=none,tablewithin=none]{caption} %подписи к объектам (рисунки, таблицы)
\usepackage{listings} %работа с листингами
\usepackage{indentfirst} %отступ первого абзаца в разделе
\usepackage{enumitem} %настройка маркированных и нумерованных списков (см. примеры настройки в тексте)
\usepackage{url} %формирование ссылок на электронные источники
\usepackage{fancyhdr} %Настройка нумерации страниц
\usepackage{tocloft} %Настройка заголовка для содержания

%====================================================================
% Мои настройки
\usepackage{amsmath}
\usepackage{amssymb}
\usepackage{booktabs}
\usepackage{caption}
\usepackage{subfig}
\usepackage{algorithm}
\usepackage{algorithmic}
\usepackage{bm}
\usepackage[para,online,flushleft]{threeparttable}
%\graphicspath{ {imgs/} }
\usepackage{xifthen}

%====================================================================

%====================================================================
%Настройки макета
%----------------
%Содержимое этого блока не должно подвергаться изменению
%====================================================================
\selectlanguage{russian}
\setlength{\parindent}{1.27cm}

%---------Размеры страницы
%\textwidth=11.3cm \textheight=17cm 
%\hoffset=-30mm \voffset=-15mm

%---------Настройка подписей к таблицам
\DeclareCaptionFormat{mplain}{#1#2\par \centering #3\par}
\captionsetup[table]{format=mplain,
justification=raggedleft,%
labelsep=none,%
singlelinecheck=false,%
skip=3pt}

%---------Настройка подписей к таблицам
%\setbeamertemplate{caption}[numbered]
%\setbeamertemplate{footline}[frame number]
\newcommand{\rulesep}{\unskip\ \vrule\ }
\newcommand{\source}[1]{\caption*{\textbf{Источник}: {#1}} }

%Настройка нумерации страниц
\fancyhf{} % clear all header and footers
\renewcommand{\headrulewidth}{0pt} % remove the header rule
\rfoot{\small \thepage}
\pagestyle{fancy}

%Настройка заголовка для содержания
\renewcommand{\cfttoctitlefont}{\hfill\normalfont\large\bfseries}
\renewcommand{\cftaftertoctitle}{\hfill\thispagestyle{empty}} 

%Настройка теорем
\theoremseparator{.}

%---------Команды рубрикации--------------

%Заголовки
\makeatletter
\renewcommand{\section}{\@startsection{section}{1}%
{\parindent}{-3.5ex plus -1ex minus -.2ex}%
{2.3ex plus.2ex}{\normalfont\large\bfseries}}

\renewcommand{\subsection}{\@startsection{subsection}{2}%
{\parindent}{-3.5ex plus -1ex minus -.2ex}%
{1.5ex plus.2ex}{\normalfont\large\bfseries}}

\renewcommand{\subsubsection}{\@startsection{subsubsection}{3}%
{\parindent}{-1.5ex plus -1ex minus -.2ex}%
{0.5ex plus.2ex}{\normalfont\bfseries}}
\makeatother

%Команда уровня главы
\newcommand{\mysection}[1]{
 \newpage
 \refstepcounter{section}
 {
  \section*{Глава \thesection. #1 \raggedright }
 }
 \addcontentsline{toc}{section}{Глава \thesection. #1} 
}

%Команда уровня параграфа
\newcommand{\mysubsection}[1]{
 \refstepcounter{subsection}
 \subsection*{\thesubsection. #1}
 \addcontentsline{toc}{subsection}{\thesubsection. #1}
}

%Команда третьего уровня
\newcommand{\mysubsubsection}[1]{
\refstepcounter{subsubsection}
% \addcontentsline{toc}{subsubsection}{\thesubsubsection. #1}
\subsubsection*{#1}
}

%Оформление Приложений
\newcounter{appendix}
\newcommand{\addappendix}[1]{
 \newpage
 \refstepcounter{appendix} 
 \section*{ПРИЛОЖЕНИЕ \theappendix. \\#1}
 \addcontentsline{toc}{section}{ПРИЛОЖЕНИЕ \theappendix. #1}
}

%Команда ненумерованной главы
\newcommand{\mynonumbersection}[1]{
\newpage
{
%	\begin{center}\section*{#1}\end{center}
	\centering\section*{#1}
}
\addcontentsline{toc}{section}{#1} 
}

%--------Настройка маркированных и нумерованных списков
\setlist{itemsep=0pt,topsep=0pt}

%--------Настройка листингов программного кода
\lstloadlanguages{C,[ANSI]C++}%!настройка листинга
%Можно подключить другие языки (см документацию к пакету)

%--------Тонкая настройка листингов
\lstset{
inputencoding=utf8x,
extendedchars=false,
showstringspaces=false,
showspaces=false,
keepspaces = true,
basicstyle=\small\ttfamily,
keywordstyle=\bfseries,
tabsize=2,                      % sets default tabsize to 2 spaces
captionpos=t,                   % sets the caption-position to bottom
breaklines=true,                % sets automatic line breaking
breakatwhitespace=true,        % sets if automatic breaks should only happen at whitespace
title=\lstname,                 % show the filename of files included with \lstinputlisting;
basewidth={0.5em,0.45em},
}

%----------Настройка подписей к листингам
\renewcommand{\lstlistingname}{Листинг}

%------------Подключение стиля для оформления списка литературы
\makeatletter
\renewcommand{\@biblabel}[1]{#1.\hfill}
\makeatother
\bibliographystyle{ugost2003s}
\PrerenderUnicode{ЙЦУКЕНГШЩЗХЪЭЖДЛОРПАВЫФЯЧСМИТЬБЮйцукенгшщзхъэждлорпавыфячсмитьбю}


%----------Use frames for text
\usepackage{mdframed}
\newcommand{\myframe}[1]{
\begin{mdframed}
\begin{center}
#1
\end{center}
\end{mdframed}
}

%----------Setup images
\captionsetup{figurename=Рисунок}

\newcommand{\myimage}[4]{ % source?; width; caption; img
\begin{figure}[h]
\centering
\includegraphics[width=#2\textwidth]{#4}
\caption{#3}
\ifthenelse{\equal{#1}{}}{
}{
\source{#1}
}
\end{figure}
}

%----------Setup Math stuff
\newtheorem{definition}{Определение}[section]
\newtheorem{proof}{Доказательство}
\newtheorem{theorem}{Теорема}

\makeatletter
\renewcommand*{\ALG@name}{Алгоритм}
\makeatother

\newcommand{\mydefinition}[1]{
\myframe{
\theoremstyle{definition}
\begin{definition}
#1
\end{definition}
}}

\newcommand{\myproof}[1]{
\theoremstyle{proof}
\begin{proof}
#1
\end{proof}
}

\newcommand{\mytheorem}[1]{
\myframe{
\theoremstyle{theorem}
\begin{theorem}
#1
\end{theorem}
}}

\newcommand{\myequation}[2][]{
\begin{equation}
\ifthenelse{\equal{#1}{}}{
\begin{aligned}#2,\end{aligned}
}{
\left.\begin{aligned}#2\end{aligned}\right\} #1
}
\end{equation}
}

\graphicspath{ {imgs/} }

\begin{document}

%==================================================
\mynonumbersection{МАТЕМАТИЧЕСКАЯ КИБЕРНЕТИКА}

%==================================================
\mysection{Введение}

\mydefinition{Кибернетика (с др.-греч. ``искусство управления'') – наука об общих закономерностях получения, хранения, преобразования и передачи информации в сложных управляющих системах, будь то машины, живые организмы или общество} 
\

Термин ``кибернетика'' изначально ввёл в научный оборот Ампер, который в своём фундаментальном труде «Опыт о философии наук, или аналитическое изложение естественной классификации всех человеческих знаний», определил кибернетику как науку об управлении государством, которая должна обеспечить гражданам разнообразные блага. В современном понимании — как наука об общих закономерностях процессов управления и передачи информации в машинах, живых организмах и обществе, термин впервые был предложен Норбертом Винером в 1948 году. \\

%==================================================
\mysection{Терминология}

\mydefinition{Динамические системы — это операторы, которые входные сигналы переводят в выходные сигналы. А сигналы — это просто функция времени, которое может быть либо дискретным, в этом случае сигнал — это последовательность значений, либо непрерывным, и тогда сигнал — это функция, заданная на полуоси.}
\

\myimage{}{0.4}{}{system}
\

Кроме входного сигнала на выходе сказывается значение вектора начального состояния. И часть выхода возможного системы — это тоже ее состояние. Размерности $x_0$ и $x(t)$ совпадают в любой момент $t$. \\

\myimage{}{0.4}{Пространство состояний}{states}
\

Динамические системы от операторов общего вида отличает принцип причинности. \\

\mydefinition{Неупреждаемость: для любого $x(0$ из равенств:
\myequation{e_1(t) = e_2(t) \forall t \in T}
следует, что
\myequation{s_1(t) = s_2(t), x_1(t) = x_2(t) \forall t \in T}
где
\myequation{\begin{bmatrix} s_{1,2}(\cdot) \\ x_{1,2}(\cdot) \end{bmatrix} = A\begin{bmatrix} e_{1,2}(\cdot) \\ x_0 \end{bmatrix}}
}
\

Состояние от других компонент выхода отличается одним важным свойством – если два разных входных сигнала привели систему в одну и ту же точку по состоянию, и после этого входные сигналы совпадают, то и выходные, начиная с этого момента, будут совпадать, причем это относится к любому выходу. Примером может служить движение материальной точки в пространстве. Ее состоянием является не только координата, но и скорость в текущий момент. \\

\mydefinition{Динамическая система линейна, если линеен задающий её оператор.}
\

\mydefinition{Динамическая система стационарна, если сдвиг входа приводит к сдвигу выхода.}
\

Если у нас есть несколько динамических систем с подходящими размерностями входов и выходов, мы можем из них комбинировать более сложные системы. Есть три типичных способа комбинировать динамические системы: \\

\begin{enumerate}
  \item последовательное соединение, когда выход одной системы служит входом для другой системы.
  \item параллельное соединение, когда один и тот же вход поступает на обе системы, а выходы их складываются.
  \item соединение типа обратной связи. Предположим, что у системы $A$ имеется два входа — $e$ и $u$, и мы $u$ формируем, как выход другой динамической системы $B$, на вход которой поступает выход $y$. Не всякая таким образом составленная система будет корректно отображать некоторое отображение из входов $e$ в выходы $y$.
\end{enumerate}
\

Обычно имеет смысл выделять некий содержательный смысл разных компонент входа и выхода. В случае входа мы имеем: \\

\begin{enumerate}
  \item внешний вход или возмущение. Например, ветер, который бросает самолет или автомобиль. 
  \item управление, управляющее воздействие
  \item это начальное состояние.
\end{enumerate}
\

Выходы, помимо того, что некоторые из них являются состоянием, а некоторые только функцией от состояния, бывают: \\

\begin{enumerate}
 \item измеряемыми, те, которые мы можем использовать при построении управляющего сигнала.
 \item регулируемые — это те, значения которых нам важны для достижения цели управления.
\end{enumerate}
\

Цели управления бывают, например, следующие: \\

\begin{itemize}
  \item стабилизация выхода около конкретного постоянного значения. 
  \item отслеживание переменного сигнала, доступного, например, измерению.
  \item минимизация некоторого функционала, заданного на выходах и входах. Пример — это перевести спутник с одной орбиты на другую с минимальным расходом горючего.
\end{itemize}
\

Для решения этих задач есть два принципиально разных подхода: \\

\begin{enumerate}
  \item программное управление. Программное управление возможно именно там, где возмущающие воздействия отсутствуют или незначительные. Имеется в виду тот же спутник. 
  \item управление обратными связями. Если внешняя среда активно влияет на объект управления, то добиться цели управления чаще всего можно только с помощью обратных связей. Схема ее такая — имеется внешние воздействия, обычно неизмеряемые, но из известного класса, есть регулируемая величина, есть измеряемый выход, который используется для генерации управляющего воздействия.
\end{enumerate}
\

%==================================================
\mysection{Примеры}

%========================================
\mysubsection{Груз на пружине}

Рассмотрим груз, подвешенный на пружине. Пусть выход $y$ — это отклонение пружины от точки равновесия. Сил в данном случае три: \\

\begin{enumerate}
  \item $u$ – управляющее воздействие, приложенное к грузу. 
  \item сила растяжения пружины пропорциональное выходу.
  \item сила вязкого трения, пропорциональная скорости движения груза.
\end{enumerate}
\

\myimage{}{0.4}{}{spring}
\

Воспользуемся вторым законом Ньютона. Масса $\times$ ускорение — это сумма всех действующих на тело сил: \\

\myequation{M \ddot{y} = u - Ky - b \dot{y} \iff M \ddot{y} + b \dot{y} + Ky = u, u(0) = y_0, \dot{y}(0) = y_1}
\

где:
\begin{itemize}
  \item $[y(t), \dot{y}(t)]$ – состояние в момент $t$
  \item начало отсчета $y$ в точке равновесия
  \item входное воздействие $u$ – приложенная к грузу сила
  \item $b$ – коэффициент вязкого трения
  \item $K$ – коэффициент упругости
\end{itemize}
\

%========================================
\mysubsection{Управляемый маятник}

Рассмотрим груз, подвешенный на тонкой нитке. $u$ – приложенная сила к грузу, $y$ – угол отклонения груза. Тогда по закону Ньютона имеем: \\

\myequation{ML \ddot{y} = -Mg \sin{y} + u}
\

\myimage{}{0.4}{}{pendulum}
\

При условии, что $|y| < \pi / 4 \implies \sin{y} \approx y$, тогда имеем: \\

\myequation{\ddot{y} + \frac{g}{L}y = \frac{1}{ML}u}
\

\end{document}

%\begin{center}
%\begin{tabular}{p{0.3\textwidth} | p{0.3\textwidth} p{0.3\textwidth} }
%& Эмпирические & Механистические \\
%\hline \\
%Детерминированные & Предсказание роста численности скота через регрессионную зависимость от потребления корма & Движение планет, основанное на Ньютоновской механике (дифференциальные уравнения) \\
%& & \\
%Стохастические & Анализ вариации различных урожаев на разных полях и в разные годы & Генетическое многообразие в маленьких популяциях, основанное на законах Менделя (вероятностные уравнения) \\
%\end{tabular}
%\end{center}
%\
%

%
