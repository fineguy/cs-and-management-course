\input{../base.tex}
\graphicspath{ {imgs/} }

\begin{document}

%==================================================
\mynonumbersection{МАТЕМАТИЧЕСКАЯ КИБЕРНЕТИКА}

%==================================================
\mysection{Введение}

\mydefinition{Кибернетика (с др.-греч. ``искусство управления'') – наука об общих закономерностях получения, хранения, преобразования и передачи информации в сложных управляющих системах, будь то машины, живые организмы или общество} 
\

Термин ``кибернетика'' изначально ввёл в научный оборот Ампер, который в своём фундаментальном труде «Опыт о философии наук, или аналитическое изложение естественной классификации всех человеческих знаний», определил кибернетику как науку об управлении государством, которая должна обеспечить гражданам разнообразные блага. В современном понимании — как наука об общих закономерностях процессов управления и передачи информации в машинах, живых организмах и обществе, термин впервые был предложен Норбертом Винером в 1948 году. \\

%==================================================
\mysection{Терминология}

\mydefinition{Динамические системы — это операторы, которые входные сигналы переводят в выходные сигналы. А сигналы — это просто функция времени, которое может быть либо дискретным, в этом случае сигнал — это последовательность значений, либо непрерывным, и тогда сигнал — это функция, заданная на полуоси.}
\

\myimage{}{0.4}{}{system}
\

Кроме входного сигнала на выходе сказывается значение вектора начального состояния. И часть выхода возможного системы — это тоже ее состояние. Размерности $x_0$ и $x(t)$ совпадают в любой момент $t$. \\

\myimage{}{0.4}{Пространство состояний}{states}
\

Динамические системы от операторов общего вида отличает принцип причинности. \\

\mydefinition{Неупреждаемость: для любого $x(0$ из равенств:
\myequation{e_1(t) = e_2(t) \forall t \in T}
следует, что
\myequation{s_1(t) = s_2(t), x_1(t) = x_2(t) \forall t \in T}
где
\myequation{\begin{bmatrix} s_{1,2}(\cdot) \\ x_{1,2}(\cdot) \end{bmatrix} = A\begin{bmatrix} e_{1,2}(\cdot) \\ x_0 \end{bmatrix}}
}
\

Состояние от других компонент выхода отличается одним важным свойством – если два разных входных сигнала привели систему в одну и ту же точку по состоянию, и после этого входные сигналы совпадают, то и выходные, начиная с этого момента, будут совпадать, причем это относится к любому выходу. Примером может служить движение материальной точки в пространстве. Ее состоянием является не только координата, но и скорость в текущий момент. \\

\mydefinition{Динамическая система линейна, если линеен задающий её оператор.}
\

\mydefinition{Динамическая система стационарна, если сдвиг входа приводит к сдвигу выхода.}
\

Если у нас есть несколько динамических систем с подходящими размерностями входов и выходов, мы можем из них комбинировать более сложные системы. Есть три типичных способа комбинировать динамические системы: \\

\begin{enumerate}
  \item последовательное соединение, когда выход одной системы служит входом для другой системы.
  \item параллельное соединение, когда один и тот же вход поступает на обе системы, а выходы их складываются.
  \item соединение типа обратной связи. Предположим, что у системы $A$ имеется два входа — $e$ и $u$, и мы $u$ формируем, как выход другой динамической системы $B$, на вход которой поступает выход $y$. Не всякая таким образом составленная система будет корректно отображать некоторое отображение из входов $e$ в выходы $y$.
\end{enumerate}
\

Обычно имеет смысл выделять некий содержательный смысл разных компонент входа и выхода. В случае входа мы имеем: \\

\begin{enumerate}
  \item внешний вход или возмущение. Например, ветер, который бросает самолет или автомобиль. 
  \item управление, управляющее воздействие
  \item это начальное состояние.
\end{enumerate}
\

Выходы, помимо того, что некоторые из них являются состоянием, а некоторые только функцией от состояния, бывают: \\

\begin{enumerate}
 \item измеряемыми, те, которые мы можем использовать при построении управляющего сигнала.
 \item регулируемые — это те, значения которых нам важны для достижения цели управления.
\end{enumerate}
\

Цели управления бывают, например, следующие: \\

\begin{itemize}
  \item стабилизация выхода около конкретного постоянного значения. 
  \item отслеживание переменного сигнала, доступного, например, измерению.
  \item минимизация некоторого функционала, заданного на выходах и входах. Пример — это перевести спутник с одной орбиты на другую с минимальным расходом горючего.
\end{itemize}
\

Для решения этих задач есть два принципиально разных подхода: \\

\begin{enumerate}
  \item программное управление. Программное управление возможно именно там, где возмущающие воздействия отсутствуют или незначительные. Имеется в виду тот же спутник. 
  \item управление обратными связями. Если внешняя среда активно влияет на объект управления, то добиться цели управления чаще всего можно только с помощью обратных связей. Схема ее такая — имеется внешние воздействия, обычно неизмеряемые, но из известного класса, есть регулируемая величина, есть измеряемый выход, который используется для генерации управляющего воздействия.
\end{enumerate}
\

%==================================================
\mysection{Примеры}

%========================================
\mysubsection{Груз на пружине}

Рассмотрим груз, подвешенный на пружине. Пусть выход $y$ — это отклонение пружины от точки равновесия. Сил в данном случае три: \\

\begin{enumerate}
  \item $u$ – управляющее воздействие, приложенное к грузу. 
  \item сила растяжения пружины пропорциональное выходу.
  \item сила вязкого трения, пропорциональная скорости движения груза.
\end{enumerate}
\

\myimage{}{0.4}{}{spring}
\

Воспользуемся вторым законом Ньютона. Масса $\times$ ускорение — это сумма всех действующих на тело сил: \\

\myequation{M \ddot{y} = u - Ky - b \dot{y} \iff M \ddot{y} + b \dot{y} + Ky = u, u(0) = y_0, \dot{y}(0) = y_1}
\

где:
\begin{itemize}
  \item $[y(t), \dot{y}(t)]$ – состояние в момент $t$
  \item начало отсчета $y$ в точке равновесия
  \item входное воздействие $u$ – приложенная к грузу сила
  \item $b$ – коэффициент вязкого трения
  \item $K$ – коэффициент упругости
\end{itemize}
\

%========================================
\mysubsection{Управляемый маятник}

Рассмотрим груз, подвешенный на тонкой нитке. $u$ – приложенная сила к грузу, $y$ – угол отклонения груза. Тогда по закону Ньютона имеем: \\

\myequation{ML \ddot{y} = -Mg \sin{y} + u}
\

\myimage{}{0.4}{}{pendulum}
\

При условии, что $|y| < \pi / 4 \implies \sin{y} \approx y$, тогда имеем: \\

\myequation{\ddot{y} + \frac{g}{L}y = \frac{1}{ML}u}
\

\end{document}

%\begin{center}
%\begin{tabular}{p{0.3\textwidth} | p{0.3\textwidth} p{0.3\textwidth} }
%& Эмпирические & Механистические \\
%\hline \\
%Детерминированные & Предсказание роста численности скота через регрессионную зависимость от потребления корма & Движение планет, основанное на Ньютоновской механике (дифференциальные уравнения) \\
%& & \\
%Стохастические & Анализ вариации различных урожаев на разных полях и в разные годы & Генетическое многообразие в маленьких популяциях, основанное на законах Менделя (вероятностные уравнения) \\
%\end{tabular}
%\end{center}
%\
%

%
