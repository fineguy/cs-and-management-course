\documentclass[14pt]{extarticle} %Класс позволяет использовать базовые шрифты бОльших размеров
\usepackage[utf8x]{inputenc} %кодировка файла макета utf8
\usepackage[russian]{babel} 
\usepackage[left=25mm,right=15mm,top=20mm,bottom=20mm]{geometry} %Попытка разобраться с полями страниц
\usepackage{ntheorem} %окружение для настройки теорем 
\usepackage{graphicx} %работа с рисунками
\usepackage[labelsep=period,figurewithin=none,tablewithin=none]{caption} %подписи к объектам (рисунки, таблицы)
\usepackage{listings} %работа с листингами
\usepackage{indentfirst} %отступ первого абзаца в разделе
\usepackage{enumitem} %настройка маркированных и нумерованных списков (см. примеры настройки в тексте)
\usepackage{url} %формирование ссылок на электронные источники
\usepackage{fancyhdr} %Настройка нумерации страниц
\usepackage{tocloft} %Настройка заголовка для содержания

%====================================================================
% Мои настройки
\usepackage{amsmath}
\usepackage{amssymb}
\usepackage{booktabs}
\usepackage{caption}
\usepackage{subfig}
\usepackage{algorithm}
\usepackage{algorithmic}
\usepackage{bm}
\usepackage[para,online,flushleft]{threeparttable}
%\graphicspath{ {imgs/} }
\usepackage{xifthen}

%====================================================================

%====================================================================
%Настройки макета
%----------------
%Содержимое этого блока не должно подвергаться изменению
%====================================================================
\selectlanguage{russian}
\setlength{\parindent}{1.27cm}

%---------Размеры страницы
%\textwidth=11.3cm \textheight=17cm 
%\hoffset=-30mm \voffset=-15mm

%---------Настройка подписей к таблицам
\DeclareCaptionFormat{mplain}{#1#2\par \centering #3\par}
\captionsetup[table]{format=mplain,
justification=raggedleft,%
labelsep=none,%
singlelinecheck=false,%
skip=3pt}

%---------Настройка подписей к таблицам
%\setbeamertemplate{caption}[numbered]
%\setbeamertemplate{footline}[frame number]
\newcommand{\rulesep}{\unskip\ \vrule\ }
\newcommand{\source}[1]{\caption*{\textbf{Источник}: {#1}} }

%Настройка нумерации страниц
\fancyhf{} % clear all header and footers
\renewcommand{\headrulewidth}{0pt} % remove the header rule
\rfoot{\small \thepage}
\pagestyle{fancy}

%Настройка заголовка для содержания
\renewcommand{\cfttoctitlefont}{\hfill\normalfont\large\bfseries}
\renewcommand{\cftaftertoctitle}{\hfill\thispagestyle{empty}} 

%Настройка теорем
\theoremseparator{.}

%---------Команды рубрикации--------------

%Заголовки
\makeatletter
\renewcommand{\section}{\@startsection{section}{1}%
{\parindent}{-3.5ex plus -1ex minus -.2ex}%
{2.3ex plus.2ex}{\normalfont\large\bfseries}}

\renewcommand{\subsection}{\@startsection{subsection}{2}%
{\parindent}{-3.5ex plus -1ex minus -.2ex}%
{1.5ex plus.2ex}{\normalfont\large\bfseries}}

\renewcommand{\subsubsection}{\@startsection{subsubsection}{3}%
{\parindent}{-1.5ex plus -1ex minus -.2ex}%
{0.5ex plus.2ex}{\normalfont\bfseries}}
\makeatother

%Команда уровня главы
\newcommand{\mysection}[1]{
 \newpage
 \refstepcounter{section}
 {
  \section*{Глава \thesection. #1 \raggedright }
 }
 \addcontentsline{toc}{section}{Глава \thesection. #1} 
}

%Команда уровня параграфа
\newcommand{\mysubsection}[1]{
 \refstepcounter{subsection}
 \subsection*{\thesubsection. #1}
 \addcontentsline{toc}{subsection}{\thesubsection. #1}
}

%Команда третьего уровня
\newcommand{\mysubsubsection}[1]{
\refstepcounter{subsubsection}
% \addcontentsline{toc}{subsubsection}{\thesubsubsection. #1}
\subsubsection*{#1}
}

%Оформление Приложений
\newcounter{appendix}
\newcommand{\addappendix}[1]{
 \newpage
 \refstepcounter{appendix} 
 \section*{ПРИЛОЖЕНИЕ \theappendix. \\#1}
 \addcontentsline{toc}{section}{ПРИЛОЖЕНИЕ \theappendix. #1}
}

%Команда ненумерованной главы
\newcommand{\mynonumbersection}[1]{
\newpage
{
%	\begin{center}\section*{#1}\end{center}
	\centering\section*{#1}
}
\addcontentsline{toc}{section}{#1} 
}

%--------Настройка маркированных и нумерованных списков
\setlist{itemsep=0pt,topsep=0pt}

%--------Настройка листингов программного кода
\lstloadlanguages{C,[ANSI]C++}%!настройка листинга
%Можно подключить другие языки (см документацию к пакету)

%--------Тонкая настройка листингов
\lstset{
inputencoding=utf8x,
extendedchars=false,
showstringspaces=false,
showspaces=false,
keepspaces = true,
basicstyle=\small\ttfamily,
keywordstyle=\bfseries,
tabsize=2,                      % sets default tabsize to 2 spaces
captionpos=t,                   % sets the caption-position to bottom
breaklines=true,                % sets automatic line breaking
breakatwhitespace=true,        % sets if automatic breaks should only happen at whitespace
title=\lstname,                 % show the filename of files included with \lstinputlisting;
basewidth={0.5em,0.45em},
}

%----------Настройка подписей к листингам
\renewcommand{\lstlistingname}{Листинг}

%------------Подключение стиля для оформления списка литературы
\makeatletter
\renewcommand{\@biblabel}[1]{#1.\hfill}
\makeatother
\bibliographystyle{ugost2003s}
\PrerenderUnicode{ЙЦУКЕНГШЩЗХЪЭЖДЛОРПАВЫФЯЧСМИТЬБЮйцукенгшщзхъэждлорпавыфячсмитьбю}


%----------Use frames for text
\usepackage{mdframed}
\newcommand{\myframe}[1]{
\begin{mdframed}
\begin{center}
#1
\end{center}
\end{mdframed}
}

%----------Setup images
\captionsetup{figurename=Рисунок}

\newcommand{\myimage}[4]{ % source?; width; caption; img
\begin{figure}[h]
\centering
\includegraphics[width=#2\textwidth]{#4}
\caption{#3}
\ifthenelse{\equal{#1}{}}{
}{
\source{#1}
}
\end{figure}
}

%----------Setup Math stuff
\newtheorem{definition}{Определение}[section]
\newtheorem{proof}{Доказательство}
\newtheorem{theorem}{Теорема}

\makeatletter
\renewcommand*{\ALG@name}{Алгоритм}
\makeatother

\newcommand{\mydefinition}[1]{
\myframe{
\theoremstyle{definition}
\begin{definition}
#1
\end{definition}
}}

\newcommand{\myproof}[1]{
\theoremstyle{proof}
\begin{proof}
#1
\end{proof}
}

\newcommand{\mytheorem}[1]{
\myframe{
\theoremstyle{theorem}
\begin{theorem}
#1
\end{theorem}
}}

\newcommand{\myequation}[2][]{
\begin{equation}
\ifthenelse{\equal{#1}{}}{
\begin{aligned}#2,\end{aligned}
}{
\left.\begin{aligned}#2\end{aligned}\right\} #1
}
\end{equation}
}

\graphicspath{ {imgs/} }

\begin{document}

%==================================================
\mynonumbersection{КОМЬЮТЕРНЫЕ СЕТИ}

%==================================================
\mysection{Передача данных}
Когда мы общаемся, мы делимся информацией. Этот обмен может быть локальным или удаленным. Локальная коммуникация между индивидами обычно происходит лицом к лицу, в то время как удаленная коммуникация происходит на расстоянии. \\

%========================================
\mysubsection{Компоненты}
Система передачи данных имеет пять компонент: \\

\begin{enumerate}
  \item \textbf{Сообщение}. Сообщение это информация (данные), которые должны быть отправлены. Популярные форматы информации включают в себя текст, числа, картинки, аудио и видео.
  \item \textbf{Отправитель}. Отправитель это устройство, которое пересылает сообщения. Это может быть компьютер, рабочая станция, видео камера и т.п.
  \item \textbf{Приемник}. Приемник это устройство, которое получает сообщения. Это может быть компьютер, рабочая станция, телевизор и т.п.
  \item \textbf{Среда передачи}. Среда передачи это физический путь, по которому сообщение идёт от отправителя к приемнику. Некоторые примеры сред передачи включают в себя витую пару проводов, коаксиальный кабель, оптоволоконный кабель и радио волны.
  \item \textbf{Протокол}. Протокол это набор правил, которые регулируют передачу данных. Он представляет из себя соглашение между коммуникационными устройствами. Без протокола два устройства могут быть соединены, но не смогут общаться, также как человек говорящий на французском не будет понят человеком, который говорит только на японском.
\end{enumerate}

%========================================
\mysubsection{Представление данных}
Информация в наше время может иметь различные формы, такие как текст, числа, картинки, аудио и видео. \\

При передаче данных текст представляется в виде последовательности битов (нулей и единиц). Различные наборы последовательностей битов были разработаны для представления текстовых символов. Каждый набор называется кодом, и процесс представления символов называется кодированием. На сегодня самой распространенной кодировочной системой является Unicode, которая использует 32 бита для представления символа или буквы используемой в любом языке мира. American Standard Code for Information Interchange (ASCII), разработанный несколько десятилетий назад в США, теперь состоит из первых 127 символов в Unicode. \\

Числа также представляются в виде последовательностей битов. Однако, кодирование, как например ASCII, не используется, вместо этого число напрямую конвертируется в двоичное представление для упрощения математических операций. \\

Картинки также представляются в виде последовательностей битов. В простейшей форме картинка состоит из матрицы пикселей (элементов изображения), где каждый пиксель является маленькой точкой. Каждому пикселю сопоставляется последовательность битов, размер и значение которой зависят от картинки. Например, чтобы отобразить четыре градации серого, можно использовать двух-битовые последовательности. Существует несколько способов кодирования цветных изображений. Один из таких методов RGB, названный так, потому что каждый цвет представляется комбинацией трёх основных цветов: красный (Red), зеленый (Green) и синий (Blue). Интенсивности каждого цвета ставится в соответствие последовательность битов. \\ 

%========================================
\mysubsection{Поток данных}
Общение между двумя устройствами может быть симплексным, полудуплексным и дуплексным. \\

\begin{itemize}
  \item Симплексное общение – общение одностороннее: одно устройство только передает сигнал, а другое устройство только принимает сигнал. Клавиатуры и традиционные дисплеи являются примерами таких устройств.
  \item Полудуплексное общение – каждое устройство может как передавать сигнал, так и принимать его, но не в одно и тоже время. Вся емкость канала полностью занимается тем устройством, которое в данный момент передает сигнал. Примером таких устройств являются рации.
  \item Дуплексное общение – каждое устройство может одновременно как передавать сигнал, так и принимать его. Либо существует два физически отдельных пути обмена данными (один для передачи и один для приема), либо емкость канала делится между исходящим и входящим сигналом.
\end{itemize}
\

%==================================================
\mysection{Сети}
Сеть это набор устройств (также называемые нодами) соединенных коммуникационными связями. Нода может быть компьютером, принтером или любым другим устройством способным на отправление и прием данных, созданных другой нодой в сети.

%========================================
\mysubsection{Распределенная обработка}
Многие сети используют распределенную обработку, в которой выполнение задачи делится между несколькими компьютерами. Вместо одной огромной машины ответственной за все аспекты обработки отдельные компьютеры обрабатывают небольшое подмножество.

%========================================
\mysubsection{Сетевые требования}
Сеть должна удовлетворять нескольким требованиям, самыми важными из которых являются производительность, надежность и безопасность: \\

\begin{itemize}
  \item \textbf{Производительность}. Производительность может быть измерена многими способами, включая время передачи и время на ответ. Время передачи это время, проведенное между запросом и ответом. Производительность сети зависит от многих факторов, включая количество пользователей, тип среды передачи, возможности сетевого оборудования и эффективность программного обеспечения. Производительность часто оценивают по двум сетевым показателям: пропускная способность и задержка. Обычно нам нужна большая пропускная способность и меньшая задержка, но эти два показателя зачастую противоположны друг другу. 
  \item \textbf{Надежность}. В дополнение к точности доставки данных надежность сети измеряется частотой ошибок; временем, которое необходимо соединению, чтобы восстановится после ошибки; и устойчивостью сети к катастрофам.
  \item \textbf{Безопасность}. Проблемы с сетевой безопасностью включают в себя защиту данных от несанкционированного доступа, защиту данных от повреждений и разработку и внедрение политик и процедур для восстановления после атак и потери данных.
\end{itemize}
\

%========================================
\mysubsection{Типы соединений}

Сеть это два и более устройства, соединенные связью. Связь это коммуникационный путь, который передаёт данные от одного устройства к другому. Для визуализации простейший способ представления связи это линия, нарисованная между двумя точками. Для осуществления общения два устройства должны каким-то образом быть соединены одной связью. Есть два способа связи: двухточечный (point-to-point) и многоточечный (multipoint). \\

\begin{itemize}
  \item \textbf{Двухточечный}. Двухточечное соединение обеспечивает выделенную связь между двумя устройствами. Вся емкость связи используется только для общения между этими двумя устройствами. Многие двухточечные соединения используют физические провода или кабели для соединения двух точек, но и другие способы, такие как микроволны или спутниковая связь, также возможны. Когда вы переключаете телевизионные каналы с помощью инфракрасного пульта, вы устанавливаете двухточечное соединение между пультом и телевизионной системой контроля.
  \item \textbf{Многоточечный}. Многоточечное соединение это соединение, при котором более двух устройств используется единую связь. При многоточечном соединении емкость канала делится между устройствами либо пространственно, либо временно. Если несколько устройств могут использовать связь одновременно, то это пространственно-общее соединение. Если же устройствам необходимо соблюдать очередность, то это временно-общее соединение.
\end{itemize}
\

%========================================
\mysubsection{Физическая топология}
Термин \textit{физическая топология} относится к способу, которым сеть устроена физически. Один или несколько устройств присоединены к линии связи; две или несколько линий связи образуют топологию. Топология сети это геометрическое представление взаимоотношений между всеми линиями связи и соединенными устройствами. Есть четыре базовые топологии: полносвязной, звезда, шина и кольцо. \\

%==============================
\mysubsubsection{Полносвязная}
В полносвязной топологии каждое устройство имеет выделенную двухточечную связь с каждым другим устройством. Если в сети имеется $N$ устройств, то в сети будет $N(N-1)$ физических симплексных связей, либо же $N(N-1)/2$ дуплексных связей. Для этого у каждого устройства должно быть $N-1$ входных/выходных портов для соединения с другими $N-1$ устройствами. \\

\myimage{}{0.4}{Полносвязная топология}{mesh}
\

Преимущества: \\
\begin{enumerate}
  \item Использование выделенных связей гарантирует, что каждое соединение может заниматься собственной нагрузкой, тем самым избавляя нас от проблем с нагрузкой, когда одна связь используется несколькими устройствами.
  \item Полносвязная топология устойчива. Если одно связь становится нестабильной, это не влияет на работу всей системы. 
  \item Полносвязная топология безопасна. Когда сообщение передается через выделенную линию связи, его может получить только тот, кому оно предназначено. Физические ограничения предотвращают остальных пользователей от получения доступа к чужим сообщениям. 
  \item Двухточечные связи делают проще процесс идентификации и изоляции ошибок. Траффик может быть перенаправлен, чтобы избежать линий связи, в которых подозреваются проблемы. Это позволяет сетевому администратору найти точное местонахождение ошибки и помогает в поиски проблем и их решений.
\end{enumerate}
\

Недостатки: \\
\begin{enumerate}
  \item Для обеспечения связи требуется использование большого количества кабелей, так как каждое устройство должно быть подсоединено к каждому другому. Это делает сложным установку и переподключение.
  \item Опять же из-за большого количества кабелей, требуется большое количество пространства и денег.
\end{enumerate}
\

Из-за этих причин полносвязная топология обычно используется в небольшом количестве случаев, например, в качестве топологии сети между главными компьютерами в гибридной сети, которая может включать в себя другие топологии. \\

%==============================
\mysubsubsection{Звезда}
При топологии типа звезда каждое устройство имеет выделенную двухточечную связь только с центральным контроллером, который называется хабом. Устройства напрямую не соединены друг с другом. В отличие от полносвязной топологии звездная топология не позволяет прямого траффика между устройствами. Контроллер выступает в качестве обменника: если одно устройство хочет отправить данные другом, оно отсылает эти данные контроллеру, который затем передает данные другому соединенному устройству. \\

\myimage{}{0.7}{Звездная топология}{star}
\

Преимущества: \\
\begin{enumerate}
  \item По сравнению с полносвязной топологией звездная топология более дешевая. Каждому устройству необходима только одна связь и один входной/выходной порт для соединения с любым количеством других устройств.
  \item Легко установить и переконфигурировать.
  \item Устойчивость. Если одна линия связи упадет, то это скажется только на ней, все другие линии связи останутся рабочими.
  \item Напрямую также как и в полносвязной топологии можно легко обнаруживать и изолировать проблемные линии связи.
\end{enumerate}
\

Недостатки: \\
\begin{enumerate}
  \item Вся сеть зависит от одной единственной ноды – хаба. Если хаб ломается, то ломается вся сеть. 
\end{enumerate}
\

%==============================
\mysubsubsection{Шина}
В предыдущих примерах описывалось двухточечное соединение. Топология в виде шины же использует многоточечное соединение. Один длинный кабель, называемый магистралью, является основой для связи всех устройств в сети. \\

\myimage{}{0.6}{Тополгия шина}{bus}
\

Преимущества: \\
\begin{enumerate}
  \item Магистраль может быть проложена наиболее эффективным способом, к которой затем будут подключаться устройства.
\end{enumerate}
\

Недостатки: \\
\begin{enumerate}
  \item Сложно переподключить или подключить новые устройства.
  \item Сложно изолировать проблемы.
  \item Проблемы с магистралью останавливают общение во всей сети.
\end{enumerate}
\

%==============================
\mysubsubsection{Кольцо}
В кольцевой топологии каждое устройство имеет выделенную двухточечную связь только с двумя другими устройствами по бокам от себя. Сигнал передается по кольцу в одном направлении, от устройства к устройству, пока он не достигнет места назначения. 

\myimage{}{0.6}{Кольцевая топология}{ring}
\

Преимущества: \\
\begin{enumerate}
  \item Легко установить и переконфигурировать. Установка или удаление устройства требует изменение только двух связей. 
  \item Сигнал всё время путешествует в сети, если какое-то устройство долго не получает сигнал, оно может сообщить об этом, что упрощает нахождение проблем в сети.
\end{enumerate}
\

Недостатки: \\
\begin{enumerate}
  \item В случае однонаправленного кольца, проблема в единственной линии связи выводит из строя всю сеть.
\end{enumerate}
\

\end{document}
