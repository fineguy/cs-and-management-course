\documentclass[14pt]{extarticle} %Класс позволяет использовать базовые шрифты бОльших размеров
\usepackage[utf8x]{inputenc} %кодировка файла макета utf8
\usepackage[russian]{babel} 
\usepackage[left=25mm,right=15mm,top=20mm,bottom=20mm]{geometry} %Попытка разобраться с полями страниц
\usepackage{ntheorem} %окружение для настройки теорем 
\usepackage{graphicx} %работа с рисунками
\usepackage[labelsep=period,figurewithin=none,tablewithin=none]{caption} %подписи к объектам (рисунки, таблицы)
\usepackage{listings} %работа с листингами
\usepackage{indentfirst} %отступ первого абзаца в разделе
\usepackage{enumitem} %настройка маркированных и нумерованных списков (см. примеры настройки в тексте)
\usepackage{url} %формирование ссылок на электронные источники
\usepackage{fancyhdr} %Настройка нумерации страниц
\usepackage{tocloft} %Настройка заголовка для содержания

%====================================================================
% Мои настройки
\usepackage{amsmath}
\usepackage{amssymb}
\usepackage{booktabs}
\usepackage{caption}
\usepackage{subfig}
\usepackage{algorithm}
\usepackage{algorithmic}
\usepackage{bm}
\usepackage[para,online,flushleft]{threeparttable}
%\graphicspath{ {imgs/} }
\usepackage{xifthen}

%====================================================================

%====================================================================
%Настройки макета
%----------------
%Содержимое этого блока не должно подвергаться изменению
%====================================================================
\selectlanguage{russian}
\setlength{\parindent}{1.27cm}

%---------Размеры страницы
%\textwidth=11.3cm \textheight=17cm 
%\hoffset=-30mm \voffset=-15mm

%---------Настройка подписей к таблицам
\DeclareCaptionFormat{mplain}{#1#2\par \centering #3\par}
\captionsetup[table]{format=mplain,
justification=raggedleft,%
labelsep=none,%
singlelinecheck=false,%
skip=3pt}

%---------Настройка подписей к таблицам
%\setbeamertemplate{caption}[numbered]
%\setbeamertemplate{footline}[frame number]
\newcommand{\rulesep}{\unskip\ \vrule\ }
\newcommand{\source}[1]{\caption*{\textbf{Источник}: {#1}} }

%Настройка нумерации страниц
\fancyhf{} % clear all header and footers
\renewcommand{\headrulewidth}{0pt} % remove the header rule
\rfoot{\small \thepage}
\pagestyle{fancy}

%Настройка заголовка для содержания
\renewcommand{\cfttoctitlefont}{\hfill\normalfont\large\bfseries}
\renewcommand{\cftaftertoctitle}{\hfill\thispagestyle{empty}} 

%Настройка теорем
\theoremseparator{.}

%---------Команды рубрикации--------------

%Заголовки
\makeatletter
\renewcommand{\section}{\@startsection{section}{1}%
{\parindent}{-3.5ex plus -1ex minus -.2ex}%
{2.3ex plus.2ex}{\normalfont\large\bfseries}}

\renewcommand{\subsection}{\@startsection{subsection}{2}%
{\parindent}{-3.5ex plus -1ex minus -.2ex}%
{1.5ex plus.2ex}{\normalfont\large\bfseries}}

\renewcommand{\subsubsection}{\@startsection{subsubsection}{3}%
{\parindent}{-1.5ex plus -1ex minus -.2ex}%
{0.5ex plus.2ex}{\normalfont\bfseries}}
\makeatother

%Команда уровня главы
\newcommand{\mysection}[1]{
 \newpage
 \refstepcounter{section}
 {
  \section*{Глава \thesection. #1 \raggedright }
 }
 \addcontentsline{toc}{section}{Глава \thesection. #1} 
}

%Команда уровня параграфа
\newcommand{\mysubsection}[1]{
 \refstepcounter{subsection}
 \subsection*{\thesubsection. #1}
 \addcontentsline{toc}{subsection}{\thesubsection. #1}
}

%Команда третьего уровня
\newcommand{\mysubsubsection}[1]{
\refstepcounter{subsubsection}
% \addcontentsline{toc}{subsubsection}{\thesubsubsection. #1}
\subsubsection*{#1}
}

%Оформление Приложений
\newcounter{appendix}
\newcommand{\addappendix}[1]{
 \newpage
 \refstepcounter{appendix} 
 \section*{ПРИЛОЖЕНИЕ \theappendix. \\#1}
 \addcontentsline{toc}{section}{ПРИЛОЖЕНИЕ \theappendix. #1}
}

%Команда ненумерованной главы
\newcommand{\mynonumbersection}[1]{
\newpage
{
%	\begin{center}\section*{#1}\end{center}
	\centering\section*{#1}
}
\addcontentsline{toc}{section}{#1} 
}

%--------Настройка маркированных и нумерованных списков
\setlist{itemsep=0pt,topsep=0pt}

%--------Настройка листингов программного кода
\lstloadlanguages{C,[ANSI]C++}%!настройка листинга
%Можно подключить другие языки (см документацию к пакету)

%--------Тонкая настройка листингов
\lstset{
inputencoding=utf8x,
extendedchars=false,
showstringspaces=false,
showspaces=false,
keepspaces = true,
basicstyle=\small\ttfamily,
keywordstyle=\bfseries,
tabsize=2,                      % sets default tabsize to 2 spaces
captionpos=t,                   % sets the caption-position to bottom
breaklines=true,                % sets automatic line breaking
breakatwhitespace=true,        % sets if automatic breaks should only happen at whitespace
title=\lstname,                 % show the filename of files included with \lstinputlisting;
basewidth={0.5em,0.45em},
}

%----------Настройка подписей к листингам
\renewcommand{\lstlistingname}{Листинг}

%------------Подключение стиля для оформления списка литературы
\makeatletter
\renewcommand{\@biblabel}[1]{#1.\hfill}
\makeatother
\bibliographystyle{ugost2003s}
\PrerenderUnicode{ЙЦУКЕНГШЩЗХЪЭЖДЛОРПАВЫФЯЧСМИТЬБЮйцукенгшщзхъэждлорпавыфячсмитьбю}


%----------Use frames for text
\usepackage{mdframed}
\newcommand{\myframe}[1]{
\begin{mdframed}
\begin{center}
#1
\end{center}
\end{mdframed}
}

%----------Setup images
\captionsetup{figurename=Рисунок}

\newcommand{\myimage}[4]{ % source?; width; caption; img
\begin{figure}[h]
\centering
\includegraphics[width=#2\textwidth]{#4}
\caption{#3}
\ifthenelse{\equal{#1}{}}{
}{
\source{#1}
}
\end{figure}
}

%----------Setup Math stuff
\newtheorem{definition}{Определение}[section]
\newtheorem{proof}{Доказательство}
\newtheorem{theorem}{Теорема}

\makeatletter
\renewcommand*{\ALG@name}{Алгоритм}
\makeatother

\newcommand{\mydefinition}[1]{
\myframe{
\theoremstyle{definition}
\begin{definition}
#1
\end{definition}
}}

\newcommand{\myproof}[1]{
\theoremstyle{proof}
\begin{proof}
#1
\end{proof}
}

\newcommand{\mytheorem}[1]{
\myframe{
\theoremstyle{theorem}
\begin{theorem}
#1
\end{theorem}
}}

\newcommand{\myequation}[2][]{
\begin{equation}
\ifthenelse{\equal{#1}{}}{
\begin{aligned}#2,\end{aligned}
}{
\left.\begin{aligned}#2\end{aligned}\right\} #1
}
\end{equation}
}

\graphicspath{ {imgs/} }

\begin{document}

%==================================================
\mynonumbersection{БАЗЫ ДАННЫХ И СУБД}

%==================================================
\mysection{Введение}
Современный мир тонет в огромном количестве постоянно появляющейся информации. В 2017 году объем мировой информации оценивался в 2.7 Зеттабайта (1 Зеттабайт = $10^3$ Эксабайта = $10^6$ Петабайта = $10^9$ Терабайта = $10^{12}$ Гигабайта). В 2019 году эта оценка выросла до 4.4 Зеттабайт. Чтобы справиться с таким количеством информации нужны соответствующие инструменты, о которых и пойдёт речь. В частности мы познакомимся с базами данных (БД) и системами управления базами данных (СУБД). \\

В широком смысле понятие "база данных" обобщается до истории любых средств, с помощью которых человечество хранило и обрабатывало данные. В узком же смысле, применяемом в современном понимании, история баз данных начинается с 1955 года, когда появилось первое программируемое оборудование обработки записей. В 1965 году была сформирована Data Base Task Group (DBTG) – рабочая группа, разработавшая в дальнейшем язык описания данных (Data Definition Language) и манипулирования данными (Data Manipulation Language). Но даже спустя более чем 50 лет общепризнанной единой формулировки не существует, поэтому приведем определение на основе международных стандартов ISO/IEC: \\

\mydefinition{
База данных – совокупность данных, хранимых в соответствии со схемой данных, манипулирование которыми выполняют в соответствии с правилами средств моделирования данных.}
\

\myimage{}{0.3}{Что такое СУБД?}{why_dbms}
\

К сожалению наличия одних только баз данных недостаточно, чтобы обеспечить удобную работу с данными. Скажем, данные, хранящиеся в БД, могут быть необходимы для работы веб сайта, который в свою очередь порождает данные, задействованные для работы мобильного приложения. Для того, чтобы облегчить запись и получение данных используются системы управления базами данных. \\

\mydefinition{
СУБД – совокупность программных и лингвистических средств общего или специального назначения, обеспечивающих управление созданием и использованием баз данных.}
\

%==================================================
\mysection{Базы данных}

%========================================
\mysubsection{Мотивация}
Рассмотрим на примере завода, для чего нам нужны БД. Допустим, вы являетесь директором завода и хотите организовать своих работников, чтобы достичь большей эффективности. Завод является довольно большой организацией, так что в любой момент времени где-то что-то происходит (многие люди записывают или получают информацию). Вам понадобится хранить информацию об: \\

\begin{itemize}
  \item людях, которые работают на вас (инженеры, менеджеры и т.д.)
  \item организации, с которыми вы взаимодействуете (поставщики, заказчики)
  \item законченные и текущие операции:
    \begin{itemize}
      \item выплаты зарплат
      \item контроль производственных процессов
      \item закупки материалов
    \end{itemize}
  \item распределение плана по цехам
  \item и т.д...
\end{itemize}
\

Вам будет необходимо делиться некоторой информацией с другими организациями, с которым вы занимаетесь бизнесом, а также защищать часть этой информации. Таким образом вам понадобится: \\

\begin{itemize}
  \item директор, замдиректора и начальники должны иметь доступ ко всей информации и иметь возможность производить множество операций (закупать необходимые материалы, организовывать поставки готовых товаров и т.д.).
  \item бухгалтеры должны иметь доступ для контроля денежных потоков (выплата премий и зарплат, оплата контрактов с партнерами).
  \item рабочие должны иметь возможность запросить необходимые им ресурсы и отчитываться о выполнение поставленных задач.
  \item полу-публичный интерфейс доступный для партнеров, через который они смогут отслеживать выполнение общих задач.
\end{itemize}
\

Сформулируем некоторые требования к информационной системе, которая бы могла удовлетворить наши запросы: \\

\begin{enumerate}
  \item \textbf{Что описывать:} какие ключевые вещи из реального мира нам нужно описать? Насколько подробно?
  \item \textbf{Как хранить данные:} можем ли мы использовать текстовые файлы: люди.txt, организации.txt, деньги.txt? Если да, то в каком виде можно записывать информацию?
  \item \textbf{Контроль над доступом:} как организовать доступ к данным, чтобы бухгалтеры знали о движениях денег на заводе, но не домашние адреса рабочих. А рабочие бы знали о производственных процессах, но не о денежных переводах.
  \item \textbf{Сбор данных:} каким образом можно получить интересующие нас данные?
  \item \textbf{Быстродействие доступа:} некоторые данные нам будут нужны мгновенно, а некоторые можно ожидать в течение длительного времени.
  \item \textbf{Атомарность:} когда бухгалтер переводит деньги из одного места в другое, нам нужны гарантии, что либо деньги взяты из места А и положены в место Б, либо ничего не произошло. В противном случае мы рискуем не досчитаться денег.
  \item \textbf{Согласованность:} в любой момент времени данные должны быть верными (например, не должно быть двух работников с одним паспортом).
  \item \textbf{Изолированность:} несколько одновременных заказов материалов не должны перезаписать друг друга.
  \item \textbf{Устойчивость:} даже если выключится компьютер, у нас должна быть возможность восстановить все данные.
\end{enumerate}
\

Взглянув на все требования, становится понятно, что попытки вручную заниматься манипуляциями с данными обречены на неудачи. К счастью, БД позволяют удовлетворить все эти запросы.

%========================================
\mysubsection{Фундаментальные концепты}
Базы данных являются микромиром в мире компьютерных наук; их изучение включает в себя: языки, теория, операционные системы, параллельное программирование, пользовательские интерфейсы, оптимизация, алгоритмы, искусственный интеллект, системный дизайн, параллельные и распределенные системы, статистика, динамическое программирование. Некоторые концепты, на которых мы заострим внимание: \\

\textbf{Представление данных} \\
Нам нужен стабильный и структурированный способ представления данных для согласованности и эффективности совместного доступа к данным. Нужные нам концепты: \\


\begin{itemize}
  \item \textbf{Модель данных:} набор конструктов (или же парадигма) описывающая организацию данных. Например, таблицы, графы, иерархии, объекты, и т.д.
  \item \textbf{Логическая схема:} описание определенных наборов данных, использующее данную модель данных.
  \item \textbf{Физическая схема:} физическая организация данных, т.е. то, как данные и метаданные лежат на дисках.
\end{itemize}
\

\myimage{}{0.7}{Логическая схема части БД завода}{schema}
\

\textbf{Декларативность запросов и обработка запросов} \\
Высокоуровневый язык для описания операций над данными. Цель заключается в том, чтобы гарантировать независимость от данных, отделяя ``что'' вы хотите сделать с данными от того, ``как'' это будет достигнуто. \\

\begin{itemize}
  \item Высокоуровневый язык для доступа к данным.
  \item Независимость от данных (логически и физически)
  \item Оптимизационные приёмы для эффективного доступа к данным.
\end{itemize}
\

\textbf{Транзакции} \\
Базовый блок для доступа и манипуляций с данными. \\

\begin{itemize}
  \item способ группировки действий, которые должны произойти атомарно (всё, либо ничего).
  \item гарантирует переход БД из одного верного состояния в другое
  \item изолирует от параллельного исполнения других действий / транзакций.
  \item восстаналиваемы в случае проблем (например, пропадет электричество).
\end{itemize}
\

%==================================================
\mysection{Системы управления базами данных}
Задачей СУБД является обеспечение библиотеки изощренных методов и стратегий для хранения, доступа и обновления данных, которые также гарантируют быстродействие, атомарность, согласованность, изолированность и устойчивость. СУБД автоматически компилирует пользовательские декларативные запросы в план исполнения (стратегия выполнения различных шагов для исполнения пользовательских запросов), ищет эквивалентные и более эффективные способы получить тот же самый результат (оптимизация запросов) и исполняет их. \\

\myimage{}{0.7}{Два эквивалентных плана}{plan}
\

%========================================
\mysubsection{Состав СУБД}
Обычно современная СУБД содержит следующие компоненты: \\

\begin{itemize}
  \item \textbf{ядро}, которое отвечает за управление данными во внешней и оперативной памяти и журнализацию;
  \item \textbf{процессор языка базы данных}, обеспечивающий оптимизацию запросов на извлечение и изменение данных и создание, как правило, машинно-независимого исполняемого внутреннего кода;
  \item \textbf{подсистему поддержки времени исполнения}, которая интерпретирует программы манипуляции данными, создающие пользовательский интерфейс с СУБД;
  \item \textbf{сервисные программы} (внешние утилиты), обеспечивающие ряд дополнительных возможностей по обслуживанию информационной системы.
\end{itemize}
\

%========================================
\mysubsection{Классификация СУБД}
В мире существует огромное разнообразие различных СУБД, их можно условно разделить: \\

\begin{itemize}
  \item По модели данных
    \begin{itemize}
      \item иерархические. Данные представляются в виде древовидной структуры.
      \item сетевые. Данные представляются в виде графа, т.е. в отличие от иерархической модели у каждой записи-потомка может быть несколько предков.
      \item реляционные. Отношения между данными опираются на математическом понятии отношение.
      \item объектно-ориентированные. Данные представляются в виде объектов, наделенных свойствами и использующие методы взаимодействия с другими объектами.
      \item объектно-реляционные. Сочетает подходы реляционных и объектно-ориентированных СУБД.
    \end{itemize}
  \item По степени распредленности
    \begin{itemize}
      \item локальные (все части локальной СУБД размещаются на одном компьютере)
      \item распределенные (части СУБД могут размещаться не только на одном, но на двух и более компьютерах)
    \end{itemize}
  \item По способу доступа к БД
    \begin{itemize}
      \item файл-серверные. СУБД располагается на каждом клиентском компьютере, а доступ к данным осуществляется через локальную сеть.
      \item клиент-серверные. СУБД располагается на сервере вместе с БД и осуществляет доступ к БД непосредственно.
      \item встраиваемые. СУБД может быть частью некоторого программного продукта, не требуя процедуры самостоятельной установки.
    \end{itemize}
\end{itemize}

\end{document}