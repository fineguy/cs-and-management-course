\documentclass[14pt]{extarticle} %Класс позволяет использовать базовые шрифты бОльших размеров
\usepackage[utf8x]{inputenc} %кодировка файла макета utf8
\usepackage[russian]{babel} 
\usepackage[left=25mm,right=15mm,top=20mm,bottom=20mm]{geometry} %Попытка разобраться с полями страниц
\usepackage{ntheorem} %окружение для настройки теорем 
\usepackage{graphicx} %работа с рисунками
\usepackage[labelsep=period,figurewithin=none,tablewithin=none]{caption} %подписи к объектам (рисунки, таблицы)
\usepackage{listings} %работа с листингами
\usepackage{indentfirst} %отступ первого абзаца в разделе
\usepackage{enumitem} %настройка маркированных и нумерованных списков (см. примеры настройки в тексте)
\usepackage{url} %формирование ссылок на электронные источники
\usepackage{fancyhdr} %Настройка нумерации страниц
\usepackage{tocloft} %Настройка заголовка для содержания

%====================================================================
% Мои настройки
\usepackage{amsmath}
\usepackage{amssymb}
\usepackage{booktabs}
\usepackage{caption}
\usepackage{subfig}
\usepackage{algorithm}
\usepackage{algorithmic}
\usepackage{bm}
\usepackage[para,online,flushleft]{threeparttable}
%\graphicspath{ {imgs/} }
\usepackage{xifthen}

%====================================================================

%====================================================================
%Настройки макета
%----------------
%Содержимое этого блока не должно подвергаться изменению
%====================================================================
\selectlanguage{russian}
\setlength{\parindent}{1.27cm}

%---------Размеры страницы
%\textwidth=11.3cm \textheight=17cm 
%\hoffset=-30mm \voffset=-15mm

%---------Настройка подписей к таблицам
\DeclareCaptionFormat{mplain}{#1#2\par \centering #3\par}
\captionsetup[table]{format=mplain,
justification=raggedleft,%
labelsep=none,%
singlelinecheck=false,%
skip=3pt}

%---------Настройка подписей к таблицам
%\setbeamertemplate{caption}[numbered]
%\setbeamertemplate{footline}[frame number]
\newcommand{\rulesep}{\unskip\ \vrule\ }
\newcommand{\source}[1]{\caption*{\textbf{Источник}: {#1}} }

%Настройка нумерации страниц
\fancyhf{} % clear all header and footers
\renewcommand{\headrulewidth}{0pt} % remove the header rule
\rfoot{\small \thepage}
\pagestyle{fancy}

%Настройка заголовка для содержания
\renewcommand{\cfttoctitlefont}{\hfill\normalfont\large\bfseries}
\renewcommand{\cftaftertoctitle}{\hfill\thispagestyle{empty}} 

%Настройка теорем
\theoremseparator{.}

%---------Команды рубрикации--------------

%Заголовки
\makeatletter
\renewcommand{\section}{\@startsection{section}{1}%
{\parindent}{-3.5ex plus -1ex minus -.2ex}%
{2.3ex plus.2ex}{\normalfont\large\bfseries}}

\renewcommand{\subsection}{\@startsection{subsection}{2}%
{\parindent}{-3.5ex plus -1ex minus -.2ex}%
{1.5ex plus.2ex}{\normalfont\large\bfseries}}

\renewcommand{\subsubsection}{\@startsection{subsubsection}{3}%
{\parindent}{-1.5ex plus -1ex minus -.2ex}%
{0.5ex plus.2ex}{\normalfont\bfseries}}
\makeatother

%Команда уровня главы
\newcommand{\mysection}[1]{
 \newpage
 \refstepcounter{section}
 {
  \section*{Глава \thesection. #1 \raggedright }
 }
 \addcontentsline{toc}{section}{Глава \thesection. #1} 
}

%Команда уровня параграфа
\newcommand{\mysubsection}[1]{
 \refstepcounter{subsection}
 \subsection*{\thesubsection. #1}
 \addcontentsline{toc}{subsection}{\thesubsection. #1}
}

%Команда третьего уровня
\newcommand{\mysubsubsection}[1]{
\refstepcounter{subsubsection}
% \addcontentsline{toc}{subsubsection}{\thesubsubsection. #1}
\subsubsection*{#1}
}

%Оформление Приложений
\newcounter{appendix}
\newcommand{\addappendix}[1]{
 \newpage
 \refstepcounter{appendix} 
 \section*{ПРИЛОЖЕНИЕ \theappendix. \\#1}
 \addcontentsline{toc}{section}{ПРИЛОЖЕНИЕ \theappendix. #1}
}

%Команда ненумерованной главы
\newcommand{\mynonumbersection}[1]{
\newpage
{
%	\begin{center}\section*{#1}\end{center}
	\centering\section*{#1}
}
\addcontentsline{toc}{section}{#1} 
}

%--------Настройка маркированных и нумерованных списков
\setlist{itemsep=0pt,topsep=0pt}

%--------Настройка листингов программного кода
\lstloadlanguages{C,[ANSI]C++}%!настройка листинга
%Можно подключить другие языки (см документацию к пакету)

%--------Тонкая настройка листингов
\lstset{
inputencoding=utf8x,
extendedchars=false,
showstringspaces=false,
showspaces=false,
keepspaces = true,
basicstyle=\small\ttfamily,
keywordstyle=\bfseries,
tabsize=2,                      % sets default tabsize to 2 spaces
captionpos=t,                   % sets the caption-position to bottom
breaklines=true,                % sets automatic line breaking
breakatwhitespace=true,        % sets if automatic breaks should only happen at whitespace
title=\lstname,                 % show the filename of files included with \lstinputlisting;
basewidth={0.5em,0.45em},
}

%----------Настройка подписей к листингам
\renewcommand{\lstlistingname}{Листинг}

%------------Подключение стиля для оформления списка литературы
\makeatletter
\renewcommand{\@biblabel}[1]{#1.\hfill}
\makeatother
\bibliographystyle{ugost2003s}
\PrerenderUnicode{ЙЦУКЕНГШЩЗХЪЭЖДЛОРПАВЫФЯЧСМИТЬБЮйцукенгшщзхъэждлорпавыфячсмитьбю}


%----------Use frames for text
\usepackage{mdframed}
\newcommand{\myframe}[1]{
\begin{mdframed}
\begin{center}
#1
\end{center}
\end{mdframed}
}

%----------Setup images
\captionsetup{figurename=Рисунок}

\newcommand{\myimage}[4]{ % source?; width; caption; img
\begin{figure}[h]
\centering
\includegraphics[width=#2\textwidth]{#4}
\caption{#3}
\ifthenelse{\equal{#1}{}}{
}{
\source{#1}
}
\end{figure}
}

%----------Setup Math stuff
\newtheorem{definition}{Определение}[section]
\newtheorem{proof}{Доказательство}
\newtheorem{theorem}{Теорема}

\makeatletter
\renewcommand*{\ALG@name}{Алгоритм}
\makeatother

\newcommand{\mydefinition}[1]{
\myframe{
\theoremstyle{definition}
\begin{definition}
#1
\end{definition}
}}

\newcommand{\myproof}[1]{
\theoremstyle{proof}
\begin{proof}
#1
\end{proof}
}

\newcommand{\mytheorem}[1]{
\myframe{
\theoremstyle{theorem}
\begin{theorem}
#1
\end{theorem}
}}

\newcommand{\myequation}[2][]{
\begin{equation}
\ifthenelse{\equal{#1}{}}{
\begin{aligned}#2,\end{aligned}
}{
\left.\begin{aligned}#2\end{aligned}\right\} #1
}
\end{equation}
}

\graphicspath{ {imgs/} }

\begin{document}

%==================================================
\mynonumbersection{ОПТИМАЛЬНОЕ УПРАВЛЕНИЕ}

%==================================================
\mysection{Введение}

%========================================
\mysubsection{Динамика}

Начнём с рассмотрения обыкновенного дифференциального уравнения (ОДУ): \\

\myequation{
\begin{cases}
\dot{x}(t) = f(x(t)) & (t > 0) \\
x(0) = x^0 & \\
\end{cases}
}
\

В данном случае у нас есть начальная точка $x^0 \in \mathbb{R}^n$ и функция $f: \mathbb{R}^n  \rightarrow \mathbb{R}^n$. Неизвестным является кривая $x: [0, \infty) \rightarrow \mathbb{R}^n$, которую мы интерпретируем как динамическую эволюцию состояния некоторой системы. \\

%========================================
\mysubsection{Контролируемая динамика}

Мы немного обобщим и предположим, что теперь функция $f$ также зависит от некоторых ``контролирующих'' параметров принадлежащих множеству $A \subset \mathbb{R}^m$, так чтобы $f: \mathbb{R}^n \times A \rightarrow \mathbb{R}^n$. Тогда, если мы выберем некоторый элемент $a \in A$ и рассмотрим динамику: \\

\myequation{
\begin{cases}
\dot{x}(t) = f(x(t), a) & (t > 0) \\
x(0) = x^0, & \\
\end{cases}
}
\

мы получим эволюцию нашей системы в ситуации, когда параметр является константой равной $a$. \\

Далее мы можем изменять значение параметра вместе с изменением системы. Определим функцию $\alpha: [0, \infty) \rightarrow A$, называемую контролем. В таком случае мы получаем следующее ОДУ: \\

\myequation{
\begin{cases}
\dot{x}(t) = f(x(t), \alpha(t)) & (t > 0) \\
x(0) = x^0, & \\
\end{cases}
}
\

при этом траекторию $x(\cdot)$ мы будем воспринимать как ответ системы. \\

%========================================
\mysubsection{Нотация}

Мы будем писать: \\

\myequation{f(x,a) = \begin{pmatrix} f^1(x,a) \\ \vdots \\ f^n(x,a) \end{pmatrix} , }
\

чтобы изобразить компоненты $f$, аналогично: \\

\myequation{x(t) = \begin{pmatrix} x^1(t) \\ \vdots \\ x^n(t) \end{pmatrix} .}
\

Мы также обозначим: \\

\myequation{\mathbb{A} = \{ \alpha: [0, \infty) \rightarrow A \, | \, \alpha(\cdot) \, \text{измерима} \} ,}
\

чтобы обозначить множество всех возможных допустимых контролей, где: \\

\myequation{\alpha(t) = \begin{pmatrix} \alpha^1(t) \\ \vdots \\ \alpha^m(t) \end{pmatrix} .}
\

Обратите внимание, что решение ОДУ $x(\cdot)$ зависит от $\alpha(\cdot)$ и начальных условий. Поэтому, строго говоря, мы бы должны были писать $x(\cdot) = x(\cdot, \alpha(\cdot), x^0)$, чтобы отразить зависимость ответа системы $x(\cdot)$ от контроля и начального значения. \\

%========================================
\mysubsection{Награда}

Наша общая задача будет определить ``наилучший'' контроль для нашей системы. Для этого нам потребуется ввести специальный функционал награды: \\

\myequation{P[\alpha(\cdot)] := \int_0^T r(x(t), \alpha(t))dt + g(x(T)) ,}
\

где $x(\cdot)$ решает ОДУ для контроля $\alpha(\cdot)$. Здесь $r: \mathbb{R}^n \times A \rightarrow \mathbb{R}$ и $g: \mathbb{R}^n \rightarrow \mathbb{R}$ заданы, и $r$ называется скользящей наградой, а $g$ – конечной наградой. Конечное время $T > 0$ также задано. \\

%========================================
\mysubsection{Постановка задачи}

Нашей целью является нахождение контроля $\alpha^*(\cdot)$, который максимизирует награду. Другими словами мы хотим, чтобы: \\

\myequation{P[\alpha^*(\cdot)] \geq P[\alpha(\cdot)] }
\

для любого контроля $\alpha(\cdot) \in \mathbb{A}$. Такой контроль $\alpha^*(\cdot)$ называется оптимальным. \\

Такая задача ставит перед нами несколько математических вопросов: \\

\begin{enumerate}
  \item Существует-ли оптимальный контроль?
  \item Как мы можем математически характеризовать оптимальный контроль?
  \item Как мы можем построить оптимальный контроль?
\end{enumerate}
\

%==================================================
\mysection{Примеры}

%========================================
\mysubsection{Контроль производства и потребления}

Допустим, мы владеем заводом, чью продукцию мы можем контролировать. Начнем с построения математической модели, положив: \\

\myequation{x(t) = \text{количество продукции, произведенной в момент времени } t \geq 0.}
\

Мы предположим, что мы потребляем какую-то долю продукции в каждый момент времени, и также повторно используем оставшуюся часть продукции. Обозначим: \\

\myequation{\alpha(t) = \text{доля повторно использованной продукции в момент времени } t \geq 0.}
\

Это будет нашим контролем, для которого есть естественные ограничения: \\

\myequation{0 \leq \alpha(t) \leq 1 \text{ в каждый в момент времени } t \geq 0.}
\

Имея такой контроль, соответствующая динамика описывается ОДУ: \\

\myequation{
\begin{cases}
\dot{x}(t) = k \alpha(t) x(t) \\
x(0) = x^0. \\
\end{cases}
}
\

константа $k > 0$ моделирует скорость роста нашего повторного использования. Рассмотрим следующий функционал награды: \\

\myequation{P[\alpha(\cdot)] := \int_0^T (1 - \alpha(t))x(t)dt.}
\

Смысл в том, что мы хотим максимизировать наше общее потребление продукции, при учёте того, что наше потребление в момент времени $t$ равно $(1 - \alpha(t))x(t)$. Эта модель соответствует нашей общей постановке для $n = m = 1$, если мы примем: \\

\myequation{\mathbb{A} = [0, 1], f(x,a) = kax, r(x, a) = (1 - a)x, g \equiv 0}
\

Как мы потом выясним, оптимальный контроль $\alpha^*(\cdot)$ задается \textbf{переключением}: \\

\myequation{\alpha^*(\cdot) = 
\begin{cases}
1 & \text{, если } 0 \leq t \leq t^* \\
0 & \text{, если } t^* < t \leq T \\
\end{cases}
}
\

%========================================
\mysubsection{Маятник}

Рассмотрим маятник, для которого: \\

\myequation{\theta(t) = \text{угол в момент времени } t.}
\

Если внешние силы отсутствуют, то у нас есть следующие уравнения движения: \\

\myequation{
\begin{cases}
\ddot{\theta}(t) + \lambda \dot{\theta}(t) + \omega^2 \theta(t) = 0 \\
\theta(0) = \theta_1, \dot{\theta}(0) = \theta_2;
\end{cases}
}
\

решением которых являются затухающие колебания, при условии $\lambda > 0$. \\

Теперь введем крутящий момент $\alpha(\cdot)$, для которого есть физическое ограничение: \\

\myequation{|\alpha| \leq 1}
\

Тогда наша динамика становится: \\

\myequation{
\begin{cases}
\ddot{\theta}(t) + \lambda \dot{\theta}(t) + \omega^2 \theta(t) = \alpha(t) \\
\theta(0) = \theta_1, \dot{\theta}(0) = \theta_2;
\end{cases}
}
\

Определим $x_1(t) = \theta(t), x_2(t) = \dot{\theta}(t)$ и $x(t) = (x_1(t), x_2(t))$. Тогда мы можем записать эволюцию нашей системы как: \\

\myequation{\dot{x}(t) = \begin{pmatrix} \dot{x}_1 \\ \dot{x}_2 \end{pmatrix} = \begin{pmatrix} \dot{\theta} \\ \ddot{\theta} \end{pmatrix} = \begin{pmatrix} x_2 \\ -\lambda x_2 - \omega^2 x_1 + \alpha(t) \end{pmatrix} = f(x, \alpha) .}
\

Мы также введём: \\

\myequation{P[\alpha(\cdot)] = -\int_0^\tau 1dt = -\tau,}
\

для: \\

\myequation{\tau = \tau(\alpha(\cdot)) = \text{первый момент, когда } x(\tau) = 0 }
\

Мы хотим максимизировать $P[\cdot]$, тем самым мы хотим минимизировать время, за которое маятник придет в спокойствие. \\

%==================================================
\mysection{Линейный оптимальный контроль}

%========================================
\mysubsection{Существование оптимального контроля}

Рассмотрим ОДУ: \\

\myequation{
\begin{cases}
\dot{x}(t) = Mx(t) + N\alpha(t) \\
x(0) = x^0,
\end{cases}
}
\

для данных матриц $M \in \mathbb{M}^{n \times n}$ и $N \in \mathbb{M}^{n \times m}$. Обозначим за $\mathbb{A}$ куб $[-1, 1]^m \subset \mathbb{R}^m$. Определим: \\

\myequation{P[\alpha(\cdot)] = -\int_0^\tau 1dt = -\tau,}
\

где $\tau = \tau(\alpha(\cdot))$ обозначает первый момент, когда ОДУ доходит до 0. Если траектория никогда не доходит до 0, то $\tau = \infty$. \\

%========================================
\mysubsection{Задача оптимального контроля}

Нам дана начальная точка $x^0 \in \mathbb{R}^n$, и мы хотим найти оптимальный контроль $\alpha^*(\cdot)$ такой, что: \\

\myequation{P[\alpha^*(\cdot)] = \max_{\alpha(\cdot) \in \mathbb{A}} P[\alpha(\cdot)] .}
\

Тогда: \\

\myequation{\tau^* = - \rho[\alpha^*(\cdot)] \text{это минимальное время для касания нуля.} }
\

\mytheorem{\textbf{Существование оптимального контроля.} Пусть $x^0 \in \mathbb{R}^n$. Тогда существует оптимальное переключение $\alpha^*(\cdot)$. }
\

%========================================
\mysubsection{Принцип максимума для оптимального контроля}

Действительно интересная практическая задача теперь понять, как вычислить оптимальный контроль $\alpha^*(\cdot)$. \\

\mydefinition{Определим $K(t,x^0)$ как множество достижимости в момент времени $t$. То есть $K(t,x^0) = \{ x^1 \, | $ существует $\alpha(\cdot) \in \mathbb{A}$, которое переключает из $x^0$ в $x^1$ в момент времени $t \}$ .}
\

Так как $x(\cdot)$ решает ОДУ, то $x^1 \in K(t,x^0)$ тогда и только тогда: \\

\myequation{x^1 = X(t)x^0 + X(t) \int_0^t X^{-1} N \alpha(s) ds = x(t)}
\

для некоторого контроля $\alpha(\cdot) \in \mathbb{A}$. \\

\mytheorem{\textbf{Геометрия множества $K$}. Множество $K(t,x^0)$ выпуклое и замкнутое.}
\

\mytheorem{\textbf{Принцип максимума Понтрягина}. Существует ненулевой вектор $h$ такой, что \myequation{h^T X^{-1}(t) N \alpha^*(t) = \max_{a \in \mathbb{A}} \{ h^T X^{-1} N a \} } для каждого момента времени $0 \leq t \leq \tau^*$.} 
\

\mytheorem{\textbf{Принцип максимума Понтрягина в сопряженной форме}. Пусть $\alpha^*(\cdot)$ это оптимальный контроль, и $x^*(\cdot)$ это соответсвующий ему ответ. Тогда существует функция $p^*(\cdot): [0, \tau^*] \rightarrow \mathbb{R}^n$ такая, что \myequation{\dot{x}^*(t) = \bigtriangledown_p H(x^*(t), p^*(t), \alpha^*(t)) } \myequation{\dot{p}^*(t) = -\bigtriangledown_x H(x^*(t), p^*(t), \alpha^*(t)) } \myequation{H(x^*(t), p^*(t), \alpha^*(t)) = \max_{a \in \mathbb{A}} H(x^*(t), p^*(t), a) }. }
\

\end{document}

%\begin{center}
%\begin{tabular}{p{0.3\textwidth} | p{0.3\textwidth} p{0.3\textwidth} }
%& Эмпирические & Механистические \\
%\hline \\
%Детерминированные & Предсказание роста численности скота через регрессионную зависимость от потребления корма & Движение планет, основанное на Ньютоновской механике (дифференциальные уравнения) \\
%& & \\
%Стохастические & Анализ вариации различных урожаев на разных полях и в разные годы & Генетическое многообразие в маленьких популяциях, основанное на законах Менделя (вероятностные уравнения) \\
%\end{tabular}
%\end{center}
%\
%

%
